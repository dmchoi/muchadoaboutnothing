%\chapter{SUMMARY AND LIMITATION OF EROSION MODEL}
%\label{sec:SUMMARYANDLIMITATIONSOFEROSIONMODELS}

\section{Summary of Model Simulation Results}
\label{sec:SummaryOftheSimulationResults}

%More can be learned from difficulties encountered \citep{jetten1999-521}.

In Part \ref{sec:RAINFALLINTENSITYCHANGESANDSOILEROSION}, \textit{Rainfall
Intensity and Erosion: Model Descriptions and Responses}, it has
been highlighted that, in order to estimate the effect of rainfall intensity
changes on future erosion, what data information we need. During the series
of investigations, we have found:
\begin{itemize}
  \item Chapter \ref{sec:EFFECTSOFTEMPORALSCALESOFSTROMDATA}:
  \begin{itemize}
      \item Temporal scales of rainfall data are closely related to the results
of runoff and soil loss modelling
      \item High resolution CLIGEN data generally yield more runoff and soil
loss
      \item Temporal scales of rainfall data affect estimations of soil loss
more than runoff
      \item Effects of the temporal data scale is greater for the summer
rainfall event (event on 4 July 2000)
      \item For the purpose of soil erosion simulation, 15-min breakpoint
rainfall data are chosen
      \item It may be suggested to use breakpoint data for the further
simulations in this research
  \end{itemize}
  \item Chapter \ref{sec:EFFECTSOFCONTINOUSANDDISCONTINUSSTORM}:
    \begin{itemize}
      \item WSGs (Within-Storm Gaps) affected runoff and soil erosion
simulations by WEPP and EUROSEM.
      \item RillGrow showed almost no changes in runoff and soil erosion
simulations.
      \item Although it was not evident to conclude whether WSGs have
positive or negative effects on runoff and soil erosion estimations, EUROSEM
simulated decreased runoff and soil loss rates with rainfall data with WSGs.
      \item Analyses of outputs from WEPP simulations revealed new problem.
      \item WEPP modifies original rainfall intensity data and simulates
erroneous results.
      \item When breakpoint data with time scales shorter than 60-min temporal
scale is used for WEPP simulations, WEPP will re-construct the rainfall data so
that original rainfall intensity information is lost.
    \end{itemize}
  \item Chapter \ref{sec:EFFECTSOFRAINFALLINTENSITYCHANGESONSOILEROSION}:
    \begin{itemize}
      \item WSIP affects the soil erosion amount.
      \item Runoff does not seem to be affected by WSIP changes.
      \item WSIP of events with high intensity have less influence on erosion
rate.
      \item Events with a constant low intensity produced dramatically less
erosion.
      \item This results is consistent with the lab experiment and modelling.
    \end{itemize}
\end{itemize}

% It is recognised that different ways of expressing rainfall intensity have the
% tenancy of desirability for erosion modelling. This is summarised in Table
% \ref{tab:DesirabilityOfDifferentWaysOfExpressingRainfallIntensity}.
% 
% \begin{table}[htbp]
%   \small
%   \centering
%       \caption{Desirability of different ways of expressing rainfall intensity}
%   \label{tab:DesirabilityOfDifferentWaysOfExpressingRainfallIntensity}
%     \begin{tabular}{ccl}
%     \toprule
%     Similarity to Reality & Desirability & Method of Rainfall Data
% Representation\\
%     \midrule
%     Dissimilar & Most & Amount, Time-to-Peak \& Peak Intensity\\
%     $\uparrow$ & $\uparrow$ &  Breakpoint Data without `no rainfall periods'\\
%     $\downarrow$ & $\downarrow$ &  Breakpoint Data with `no rainfall periods'\\
%     Similar & Least & Tipping Bucket Data\\
%     \bottomrule
%     \end{tabular}
% \end{table}

The effects of rainfall intensity changes on runoff and soil erosion found in
Part \ref{sec:RAINFALLINTENSITYCHANGESANDSOILEROSION} are summarised in Table
\ref{tab:SimulationSummary}.

\begin{sidewaystable}[htbp]
% \small
  \centering
  \caption{Summary of the effect of intra-storm characteristics on runoff and
soil erosion}
  \label{tab:SimulationSummary}
    \begin{tabular}{llcccc}
    \toprule
    & & \multicolumn{3}{c}{Erosion Model} & Measurement \\
    \cmidrule(rl){3-5} \cmidrule(rl){6-6}
    Intensity Pattern & & WEPP$^1$ & EUROSEM$^1$ & RillGrow &
\citet{parsons2006-68}$^2$\\
    \midrule
    \midrule
    Constant & Runoff & $-$ & $-$ & $-$ & $-$ \\
             & Soil loss & $-$ & $-$ & $-$ & $-$ \\
    \midrule
    Increasing$^\dagger$ & Runoff & $-$ & $-$ & $-$ &
$\blacktriangle\blacktriangle$ \\
                         & Soil loss &
$\blacktriangle\blacktriangle\blacktriangle$ & $\triangledown\triangledown$
& $\blacktriangle\blacktriangle$ &
$\blacktriangle\blacktriangle\blacktriangle\blacktriangle\blacktriangle$ \\
    \midrule
    Decreasing$^\dagger$ & Runoff & $-$ & $\triangledown$ & $-$ &
$\blacktriangle\blacktriangle\blacktriangle\blacktriangle$\\
                         & Soil loss &
$\blacktriangle\blacktriangle\blacktriangle$ & $\triangledown$ &
$\blacktriangle\blacktriangle\blacktriangle\blacktriangle$ &
$\blacktriangle\blacktriangle\blacktriangle\blacktriangle\blacktriangle$ \\
    \midrule
    Increasing-Decreasing$^\dagger$ & Runoff & $-$ & $\triangledown$ & $-$ &
$\blacktriangle\blacktriangle$ \\
                                    & Soil loss &
$\blacktriangle\blacktriangle\blacktriangle\blacktriangle$ &
$\triangledown\triangledown$ & $\blacktriangle\blacktriangle\blacktriangle$
& $\blacktriangle\blacktriangle\blacktriangle\blacktriangle\blacktriangle$ \\
    \midrule
    \midrule
    Continuous & Runoff & $-$ & $-$ & $-$ & n/a \\
               & Soil loss & $-$ & $-$ & $-$ & n/a \\
    \midrule
    Discontinuous$^\ddagger$ & Runoff & $\triangledown\triangledown$ &
$\triangledown\triangledown$ & $-$ & n/a \\
                             & Soil loss & $\blacktriangle$&
$\triangledown\triangledown$ & $-$ & n/a\\
    \bottomrule
    %\addlinespace[1mm]
    \multicolumn{6}{p{16cm}}{\scriptsize
    $^1$ Mean runoff and soil loss rate at the average intensity of 60 mm/hr and
10 mm/hr;
    $^2$ Runoff and soil loss rate for Constant Intensity was measured on a
experiment plot filled with sandy loam at an average intensity of 93.9 mm/hr;
    $^\dagger$ Magnitude of changes in comparison with Constant Intensity;
    $^\ddagger$ Magnitude of changes in comparison with Continuous Intensity;
    $-$: unchanged or $\geq$1\% changed;
    $\blacktriangle$: 1$<\Delta\geq$5\% increase;
    $\blacktriangle\blacktriangle$: 5$<\Delta\geq$15\% increase;
    $\blacktriangle\blacktriangle\blacktriangle$: 15$<\Delta\geq$30\%
increase;
    $\blacktriangle\blacktriangle\blacktriangle\blacktriangle$:
30$<\Delta\geq$40\% increase;

$\blacktriangle\blacktriangle\blacktriangle\blacktriangle\blacktriangle$:
$>$40\% increase;
    $\triangledown$: 1$<\Delta\geq$5\% decrease;
    $\triangledown\triangledown$: 5$<\Delta\geq$15\% decrease;
    $\triangledown\triangledown\triangledown$: 15$<\Delta\geq$30\% decrease;
    $\triangledown\triangledown\triangledown\triangledown$:
30$<\Delta\geq$50\% decrease;
    $\triangledown\triangledown\triangledown\triangledown\triangledown$:
$>$40\% decrease}\\
    \end{tabular}
\end{sidewaystable}

\section{Limitations of Erosion Models}
\label{sec:LimitationsOfErosionModels}

In CLIGEN, the rainfall duration is an artificial abstraction that is the
duration, that is a composite rainfall event with a triangular shape, calculated
by summing up all the rainfall that occurred in 24 hours. CLIGEN's rainfall
`duration' is also not a realistic concept for rainstorms, which last for more
than 24 hours. The unrealistic definition of CLIGEN's rainfall duration can
therefore be prone to unrealistic simulation of soil erosion. CLIGEN data
assumes that there is only one peak per storm. Each storm starts and ends within
a 24 hour period. This means all the storms are daily.

It is thus suggested that predictions of soil erosion may be improved if we
consider rainfall as an event-by-event rather than on a daily basis. In
other words, rather than taking a whole wet-day (24 hours) as one ``event'',
we may need to seek a way of separating rainfall events independent of the
day. In this way, soil erosion estimations for the area with dominantly low
rainfall intensity may be improved.

%Although erosion models used in this research requires high resolution
%rainfall data, such kind of data is not readily available, and seldom is
%complete and long term. This is because it is usually aggregated into other
%scales for ease of storage and portability.

%Suggestions on possible improvements of WEPP and EUROSEM for rainfall
%intensity

%Discussion about the assumption of daily rainfall = single storm---pros \&
%cons for future soil erosion estimation.

%$\filledstar\filledstar\filledstar$ Unmatched disaggregated rainfall data
%by WEPP:
%
%WEPP is doing something weird.
% When time intervals shorter than an hour were used for WEPP simulations,
% WEPP reconstructs breakpoint data from original breakpoint data to ``WEPP
% interpreted'' breakpoint data, which has different intensity information.
% The total number of breakpoint are the same so as the accumulated amount.
% However, because the time increments are changed by WEPP, rainfall intensity
% information for the original breakpoint data is not the same.
% This means that:
% \begin{enumerate}
%   \item Resulting runoff and soil loss may be estimated unrealistically,
% considering rainfall intensity is different from the observation;
%   \item Studies on impacts of rainfall intensity change on soil erosion with
% rainfall data shorter than hourly may result in runoff and soil loss
% estimations which is different from original rainfall intensity.
% \end{enumerate}
% Using breakpoint data shorter than hourly will result in feeding in wrong
% rainfall intensity information to WEPP.
% This problem seems important because it will consequently alter the original
% rainfall intensity even if we use breakpoint data in order to keep the
% original rainfall intensity unchanged. This is because breakpoint rainfall
% data are better representations of the real rainfall intensity than CLIGEN
% rainfall data (See Chapter \ref{sec:EFFECTSOFTEMPORALSCALESOFSTROMDATA})

WEPP usually requires four stages of rainfall data conversion in order to
simulate runoff and soil loss:
\begin{enumerate*}
  \item Starting with the original time step rainfall data
  \item Converting to an aggregated time stepped rainfall data by removing no
rainfall periods
  \item Parametrising the rainfall data into amount, duration, time to peak and
peak intensity
  \item Finally, regenerating disaggregated rainfall data based on the
parameters
\end{enumerate*}

After each stage, original rainfall intensity information is lost and distorted
as this research has shown.
When the details are lost, it is prone to lead to wrong simulation results. It
is paramount to maintain intensity details such as the number of intensity peaks
that occur during a storm period regardless of frequency in order to study how
these intensity peaks might affect the erosion process. Assuming just one peak
per storm is also a rather crud way of dealing rainfall intensity changes within
a storm.

It seems that the unit conversion error is a very common problem for soil models
and therefore need to be closely monitored. A clear statement of what unit is
used for the specific parameter is very important. Imperial and metric units
should not be used concurrently in any case. It is a simple mistake but can
cause seriously erroneous estimates of runoff and soil erosion.

WEPP and EUROSEM do not consider temporal variations in erodibility during a
rainfall storm. \citet{kinnell2005-2815} also pointed out this problem with WEPP
and EUROSEM. In the case of raindrop-impact-induced erosion, current so-called
process-based erosion models appear to represent the process involved
inadequately in some respects because the process involved in detachment and
transport of soil from the surface during experiments leading to model
parametrization is unknown \citep{kinnell2005-2815}.

%\citet{Parsons2006-68} found that the constant-intensity storm caused 75\%
%less erosion than the average value for the variable-intensity storms.
%
%``The assumption that a given rainfall intensity falling on a given soil
%for a given amount of time will result in a given amount of runoff and
%erosion is unsound.'' \citep{Parsons2006-68}
%
%``\ldots failing to take account of the context within which a particular
%rainfall intensity occurs and that parametrising equations using
%constant-intensity rainstorms may lead to errors in the prediction of
%interrill soil erosion.'' \citep{Parsons2006-68}
%
%``Models (USLE and WEPP) that derive interrill soil erosion directly from
%rainfall intensity can be expected to perform poorly in predicting soil
%erosion from storms exhibiting temporal variability in rainfall intensity
%as is characteristic of many runoff-producing storm events.''
%\citep{Parsons2006-68}
%
%``Other models (EUROSEM and LISEM) modulate the effect of rainfall
%intensity on erosion by assuming that detachment is controlled both by
%rainfall intensity and runoff depth and that erosion is further controlled
%by interrill transport capacity.'' \citep{Parsons2006-68}
%
%``Neither models that relate interrill soil erosion directly to rainfall
%intensity, nor those that modulate the effect of rainfall intensity by
%runoff depth and transport capacity can account for the observed effects of
%storm pattern on soil erosion.'' \citep{Parsons2006-68}

%In this section, WEPP's possible problem with `re-disaggregation' of
%original breakpoint data that are shorter than hourly should be included.
%
%$\downarrow$ this seems a repeat of what has been said before, but it need
%to be clarified that the second paragraph is true for breakpoint data.
%These problems may only be the cases for CLIGEN-generated data, not
%breakpoint data.
%
%$\filledstar\filledstar\filledstar$ WEPP generally disaggregates CLIGEN
%rainfall data into a form of breakpoint data with 10 breakpoints for
%erosion simulations. This is only true for storms that last over 60-min.
%When a storm lasts less than 60-min, WEPP disaggregates the rainfall data
%into less than ten breakpoints.
%
%$\filledstar\filledstar\filledstar$ This is a problem for BP data of 1-min
%rainfall events that last shorter than 60-min. Even if the 1-min rainfall
%has several breakpoints, it could be represented as, in a worst case, a
%single peak of rainfall which has constant (average) rainfall intensity for
%the storm duration.
