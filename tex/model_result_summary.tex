%\chapter{SUMMARY OF MODEL SIMULATION RESULTS}
%\label{sec:SUMMARYANDLIMITATIONSOFEROSIONMODELS}





%\section{Limitations of Erosion Models}
%\label{sec:LimitationsOfErosionModels}



%Although erosion models used in this research requires high resolution
%rainfall data, such kind of data is not readily available, and seldom is
%complete and long term. This is because it is usually aggregated into other
%scales for ease of storage and portability.

%Suggestions on possible improvements of WEPP and EUROSEM for rainfall
%intensity

%Discussion about the assumption of daily rainfall = single storm---pros \&
%cons for future soil erosion estimation.

%$\filledstar\filledstar\filledstar$ Unmatched disaggregated rainfall data
%by WEPP:
%
%WEPP is doing something weird.
% When time intervals shorter than an hour were used for WEPP simulations,
% WEPP reconstructs breakpoint data from original breakpoint data to ``WEPP
% interpreted'' breakpoint data, which has different intensity information.
% The total number of breakpoint are the same so as the accumulated amount.
% However, because the time increments are changed by WEPP, rainfall intensity
% information for the original breakpoint data is not the same.
% This means that:
% \begin{enumerate}
%   \item Resulting runoff and soil loss may be estimated unrealistically,
% considering rainfall intensity is different from the observation;
%   \item Studies on impacts of rainfall intensity change on soil erosion with
% rainfall data shorter than hourly may result in runoff and soil loss
% estimations which is different from original rainfall intensity.
% \end{enumerate}
% Using breakpoint data shorter than hourly will result in feeding in wrong
% rainfall intensity information to WEPP.
% This problem seems important because it will consequently alter the original
% rainfall intensity even if we use breakpoint data in order to keep the
% original rainfall intensity unchanged. This is because breakpoint rainfall
% data are better representations of the real rainfall intensity than CLIGEN
% rainfall data (See Chapter \ref{sec:EFFECTSOFTEMPORALSCALESOFSTROMDATA})

WEPP usually requires four stages of rainfall data conversion in order to
simulate runoff and soil loss:
\begin{enumerate*}
  \item Starting with the original time step rainfall data
  \item Converting to an aggregated time stepped rainfall data by removing no
rainfall periods
  \item Parametrising the rainfall data into amount, duration, time to peak and
peak intensity
  \item Finally, regenerating disaggregated rainfall data based on the
parameters
\end{enumerate*}

After each stage, original rainfall intensity information is lost and distorted
as this research has shown.

WEPP and EUROSEM do not consider temporal variations in erodibility during a
rainfall storm. \citet{kinnell2005-2815} also pointed out this problem with WEPP
and EUROSEM. In the case of raindrop-impact-induced erosion, current so-called
process-based erosion models appear to represent the process involved
inadequately in some respects because the process involved in detachment and
transport of soil from the surface during experiments leading to model
parametrization is unknown \citep{kinnell2005-2815}.

%\citet{Parsons2006-68} found that the constant-intensity storm caused 75\%
%less erosion than the average value for the variable-intensity storms.
%
%``The assumption that a given rainfall intensity falling on a given soil
%for a given amount of time will result in a given amount of runoff and
%erosion is unsound.'' \citep{Parsons2006-68}
%
%``\ldots failing to take account of the context within which a particular
%rainfall intensity occurs and that parametrising equations using
%constant-intensity rainstorms may lead to errors in the prediction of
%interrill soil erosion.'' \citep{Parsons2006-68}
%
%``Models (USLE and WEPP) that derive interrill soil erosion directly from
%rainfall intensity can be expected to perform poorly in predicting soil
%erosion from storms exhibiting temporal variability in rainfall intensity
%as is characteristic of many runoff-producing storm events.''
%\citep{Parsons2006-68}
%
%``Other models (EUROSEM and LISEM) modulate the effect of rainfall
%intensity on erosion by assuming that detachment is controlled both by
%rainfall intensity and runoff depth and that erosion is further controlled
%by interrill transport capacity.'' \citep{Parsons2006-68}
%
%``Neither models that relate interrill soil erosion directly to rainfall
%intensity, nor those that modulate the effect of rainfall intensity by
%runoff depth and transport capacity can account for the observed effects of
%storm pattern on soil erosion.'' \citep{Parsons2006-68}

%In this section, WEPP's possible problem with `re-disaggregation' of
%original breakpoint data that are shorter than hourly should be included.
%
%$\downarrow$ this seems a repeat of what has been said before, but it need
%to be clarified that the second paragraph is true for breakpoint data.
%These problems may only be the cases for CLIGEN-generated data, not
%breakpoint data.
%
%$\filledstar\filledstar\filledstar$ WEPP generally disaggregates CLIGEN
%rainfall data into a form of breakpoint data with 10 breakpoints for
%erosion simulations. This is only true for storms that last over 60-min.
%When a storm lasts less than 60-min, WEPP disaggregates the rainfall data
%into less than ten breakpoints.
%
%$\filledstar\filledstar\filledstar$ This is a problem for BP data of 1-min
%rainfall events that last shorter than 60-min. Even if the 1-min rainfall
%has several breakpoints, it could be represented as, in a worst case, a
%single peak of rainfall which has constant (average) rainfall intensity for
%the storm duration.
