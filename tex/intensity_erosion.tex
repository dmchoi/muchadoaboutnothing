\section{Soil Erosion and Rainfall Intensity}
\label{sec:RainfallIntensityAndSoilErosion}

This section provides examples of some notable erosion events which are
documented in selected publications.

\paragraph{South Downs, East Sussex, UK, October 1987
\citep{boardman1988-333}}
\label{sec:SouthDownsOctober1987}

Heavy rainfall on 7 October 1987 and subsequent storms resulted in soil losses
over 50 m$^3$/ha\ on several fields and over 200 m$^3$/ha\ on one field in the
eastern South Downs \citep{boardman1988-333}. Monthly rainfall totals at
Southover, Lewes, were 54.3 mm for September and 270.9 mm for October 1987.
Rainfall recorded at Southover, Lewes, on 7 October 1987 was 50.2 mm with a
maximum short period intensity of 6.7 mm/h\ for 5.5 hours including 40 mm/h\ for
15 minutes.

Substantial rills or gullies were formed by the rainfall event on 7 October
1987. As a result of this, following rainfalls as low as 7 mm caused runoff and
erosion \citep{boardman1988-333}.
Although there are no event-by-event records available for soil losses, it is
evident that the rainfall on 7 October 1987 played an important role, by
contributing to rill or gully generation, on  soil erosion in the area. However,
the main factors responsible for the severe erosion were land use and farming
practices.

%%%%%
\paragraph{Vicrello, Tuscany, Italy, May 1994 \citep{torri1999-131}}
\label{sec:VicrelloVolterraTuscany}

A rainfall depth of 77.8 mm fell on a field plot with a bare soil in Vicrello,
Tuscany, Italy \citep{torri1999-131}. The storm lasted for over 28 hours and
caused a soil loss of 126.2 t/ha. Maximum intensity averaged over 10 minute was
120 mm/h.

\paragraph{Hadspen, Somerset, UK, May 1998 \citep{clark2000-17}}
\label{sec:HadspenSomersetUK}
Total rainfall amount of 47.6 mm fell in Hadspen, Somerset, UK on 13 May 1998
\citep{clark2000-17}. Most rain fell between 2115 GMT to 2130 GMT reaching
rainfall intensity of $>$100 mm/h. In Nettlecombe Hill and Higher Hadspen,
ploughed fields on slopes with 2--11\textdegree\ eroded at the rates of 1.412
tonnes/m$^3$ and 1.312 tonnes/m$^3$, respectively. Total soil loss from two area
was 72.1 tonnes.

\paragraph{Ashow, Warwickshire, UK, August 1999 \citep{harrison1999-143}}
\label{sec:AshowWarwickshireUK}
On 20 August 1996 in Ashow, Warwickshire, the storm commenced at 1930 BST.
Rainfall intensity was low until 2030 BST when 24.5 mm of rain fell in 30 min
and a total of 33.5 mm fell before midnight.

One of two fields in the catchment was planted with oilseed rape eight days
before the storm. The field was ploughed and power-harrowed, and then seed
drilled with a low ground pressure buggy. It was subsequently rolled by a
tractor with low ground pressure tyres. The other field was harvested of wheat
and barley, and then rough ploughed, the soil clods being broken up using
rotating discs.

Extensive erosion of top soil occurred, followed by the development of gullies
and rills by overland flow during the storm. Approximately 790 t of sediment was
eroded from the two fields excluding the sediment that reached nearby river
(River Avon, UK). Average sediment yields was 49.7 t/ha which is equivalent to
the average ground lowering of 3.8 mm.

\paragraph{Northern Ethiopia Highlands, 1998-2000 \citep{nyssen2005-172}}
\label{sec:NorthernEthiopiaHighlands}

Rainfall intensity in Northern Ethiopia Highlands was monitored using a tipping
bucket rain gauge during 1998-2000 \citep{nyssen2005-172}. Overall rain
intensity in the area is low. 88\% of total rain volume falls with an intensity
$<$30 mm/h. Most storms have a low intensity with a brief high intensity part.
This high intensity can be observed at the beginning, in the middle or at the
end of the storm.
Although area-averaged intensity was low in this area, it was found that maximum
rain intensity at individual locations exceeded by far the threshold values for
excessive rain (see Table 5, \citealp{nyssen2005-172}). Rainfall intensities
beyond these thresholds were known to cause $>$50\% of total soil losses
\citep{krauer1988-rainfallerosivityand}. Large rain erosivity in the area is due
to larger median volume drop diameters ($D_{50}$) than those reported for other
regions of the world, rather than due to high intensity.

%%%%%
\paragraph{South Downs, East Sussex, UK, October 2000
\citep{boardman2001-346}}
\label{sec:SouthDownsOctober2000}

Exceptional rainfall in October and November 2000, especially a 24-hour fall of
about 100 mm, led to extensive erosion and property damage
\citep{boardman2001-346}. The rainfall was typical of frontal, low-intensity
events that usually occur in British winters but it lasted for a longer period
than usual period. In a 24-hour period prior to 09:00 on 12 October (i.e.\ 11
October rainday), a total rainfall of 89.9 mm was recorded
\citep{boardman2001-346}.
In a 10-hour period of continuous rainfall (23:00--09:00) 63.8 mm fell with a
maximum intensity of 11.4 mm/h\ and a maximum short-period intensity of 3.6
mm/min\ (i.e.\ 216 mm/hr) \citep{boardman2001-346}.

Rainfall of 100 mm in 24 hours has a return period of well over 100 years and a
intensity of 11 mm/h is to be expected every year \citep{boardman2001-346}. This
means that the rainfall on 11 October 2000 has a rainfall intensity that is
commonly observed in the area, but the total amount and duration are very
unlikely in the area. It is noted that high intensity rainfall within prolonged
low intensity rainfall at the time of year when the agricultural land is most
vulnerable may result in extensive erosion events.




