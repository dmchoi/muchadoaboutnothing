%\linenumbers*
\chapter{CONCLUSION}
\label{sec:CONCLUSION}

The main objective of this research has been to investigate possible
implications of climate change for future erosion with reference to rainfall
intensity changes by analysing the response of erosion models to arbitrary
rainfall intensity changes, and implicitly the process understanding on which
the models are based.
Thus, this research is a step towards the ultimate goal of predicting future
rates of soil erosion caused by future rainfall intensity changes.

To achieve this aim, following four research questions have been tackled:
\begin{enumerate}[{Question} 1.]
  \item What role does rainfall intensity play in the process descriptions which
comprise erosion models?
  \item Assuming that we use a model to predict erosion rates under the future
climate which may have different rainfall intensities from the present, what
information do we need to make predictions in terms of both climate and process
understanding?
  \item What do we know about future rainfall intensity? Does any trend exist in
the present and past rainfall intensity at the study site? If so, is the trend
consistent with future rainfall intensity predicted by climate models? If not,
what must be done to estimate future rainfall intensity?
  \item Are we in a position to predict soil erosion under the future climate?
If not, what must be done?
\end{enumerate}

% \begin{enumerate}%[Q \#1.]
%   \item What do we know about present and past rates of rainfall intensity at
%the study site, and elsewhere? Are there any obvious trends?
%   \item What role does rainfall intensity play in the process descriptions
%which comprise erosion models?
%   \item What information would we need, with respect to both climate and
%process understanding, in order to predict rates of erosion under a future
%changed climate (in particular, under a future climate which has different
%rainfall intensities from the present)?
%   \item Are we in a position to predict erosion rates under future climates,
%with different rainfall intensities from the present? If not, what must be
%done?
% \end{enumerate}

Findings with respect to each of these research questions are summarised below,
along with limitations which have resulted from the approach used. This section
finishes with some suggestions for future research.

\section{Rainfall Intensity and Process Descriptions}
\label{sec:RainfallIntensityAndProcessDescriptions}
\subsection{Findings}
\label{sec:FindingsStage1}
%***to do

There is a difference between the theoretical treatment of rainfall
intensity in erosion models, and the practical handling of it.
%*** NEED TO DEFINE ``STORM''
A first finding was that WSIV affects the results of simulations. Rainfall
events with a constant intensity produced dramatically less erosion amounts
compared with storms with the same total volume of rainfall, but increasing or
decreasing intensity. This a constant intensity is not a good representation of
rainfall intensity for a natural rainfall event as far as erosion modelling is
concerned.

In addition, it appears that a higher temporal resolution is needed for rainfall
data that have high within-storm intensity variations. WSIP have a larger effect
on estimations of soil erosion compared with runoff. Thus, the temporal
resolution of rainfall data affects the results of erosion simulations. As the
temporal resolution of rainfall increased, then the amounts of estimated soil
loss also increased. Thus a study which uses rainfall data with different
temporal resolutions will give rise to artefacts i.e.\ changes in estimated soil
loss may be due only to the different temporal resolutions of rainfall.

In addition, WSG (Within-Storm Gap) also have a marked effect on erosion
estimations. Discontinuous storms (i.e.\ with within-storm gaps) would be
expected to produce less runoff and soil loss than an equivalent continuous
storm (i.e.\ a storm with the same total rainfall, but with WSGs removed), since
the longer duration of the discontinuous storm gives rise to a lower average
intensity for the storm, compared with the continuous storm.
%*** COMPARE E.G. USLE R FACTOR HERE, ALSO OTHER REFS.
However, an unexpected result from WEPP was that more soil loss was estimated
with discontinuous rainfall than with volume-equivalent continuous rainfall.
This suggests a design flaw in the rainfall data description used in WEPP, due
to WEPP's handling of the  original breakpoint data (it is internally
reconstructed to ``WEPP-interpreted'' data, and this ``reconstructed'' data is
used in the simulations. Limitations here mean that some information is lost
when the reconstructed data is compared to the original breakpoint data.)


\subsection{Limitations}
\label{sec:LimitationsStage1}
%***to do

All data used here was for only one site (i.e.\ Woodingdean, South Downs, UK).
Thus, there is a chance that the erosion models may give different results when
other slope data are used.

%*** IMPROVE THIS
No comparisons of estimated runoff and erosion rates against measured runoff and
erosion rates were possible because of the absence of observed data for the
individual events considered. However, the changes in runoff and erosion rates
from laboratory experiments with similar storm intensity conditions show similar
runoff and erosional responses to the intensity changes. Proper comparisons of
soil erosion can only be made when there is observed erosion rates, which then
can be used as a reference. The validity of the results thus need to be
supported by a different approach.


\section{What do we need to predict erosion under future changed rainfall
intensities?}
\label{sec:WeNeedToPredictErosionUnderFutureChangedRainfallIntensities}
\subsection{Findings}
\label{sec:FindingsStage2}
%The improvement of CLIGEN has clear implications for erosion simulations.
%Updated CLIGEN is more sensitive to the changes in rainfall intensity (i.e.\
%MX~.5P) than the old one. This sensitivity change affects erosion estimations
%more considerably for the regions where low intensity rainfall events are
%dominant in comparison to the regions where high intensity rainfall events are
%dominant.
Temporal resolution of rainfall data affect results of erosion simulations. As
temporal resolutions increase, amounts of erosion estimations increase. Using
rainfall data with an inconsistent temporal resolution for erosion simulations
may result in erroneous erosion estimations. Higher temporal scale is needed
when rainfall data that have high WSIV (Within-Storm Intensity Variation) are
used for erosion estimations.
%higher than 15-min may lead to a greater increase in runoff and soil loss than
%that of longer scales (***This statement need to be changed).

Within-Storm Intensity Pattern (WSIP) have an effect on estimations of soil
erosion. WSIPs have more effects on soil loss than runoff. Rainfall events with
a constant intensity produced dramatically less erosion amounts than other WSIPs
such as increasing or decreasing. Constant intensity is not a good
representation of rainfall intensity for a natural rainfall event as far as
erosion modelling is concerned.

Within-Storm Gap (WSGs) have an effect on erosion estimations. Discontinuous
rainfall events (i.e.\ WSGs included) were expected to produce less runoff and
soil loss than continuous rainfall events (i.e.\ WSGs removed) that has the same
rainfall amount as discontinuous events. This is because relatively longer
rainfall durations of discontinuous rainfall events imply that the events have
relatively less intense intensity than continuous events. However, unexpected
results from WEPP simulations found that WEPP estimates more soil loss with
discontinuous rainfall than with continuous rainfall. This revealed a design
flaw in the rainfall data description used in WEPP. WEPP modifies original
intensity information of breakpoint data by reconstructing the data to
``WEPP-interpreted'' data, and uses this ``reconstructed'' data for simulations
of erosion.

To predict future soil erosion that has been affected by future rainfall
intensity changes, the following are required:
\begin{enumerate}
  \item Long-term rainfall data with adequately high temporal resolution, high
enough to capture details of Within-Storm Intensity Variations (WSIVs)
  \item Information about Within-Storm Intensity Patterns (WSIPs)
  \item Duration and frequency of Within-Storm Gaps (WSGs) within a storm
  \item An erosion model that can make proper use of rainfall intensity
information stated above (e.g. using data with up to 1440 breakpoints, at
least)
  \item An erosion model that can simulate long-term soil erosion
continuously---a continuous simulation model.
\end{enumerate}

%*** NEEDS TO BE IMPROVED
To predict future soil erosion that has been affected by future rainfall
intensity changes, the following are required:
\begin{enumerate}
 \item Long-term rainfall data with adequately high temporal resolution, high
enough to capture details of within-storm intensity variations
 \item Information about the duration and frequency of no-rain periods within a
storm
%*** HOW DETAILED? I.E. MINIMUM SIZE OF NO-RAIN GAP WHICH MUST BE KNOWN?
 \item An erosion model that can make proper use of this rainfall intensity
information i.e. which does not simply the breakpoint data as does WEPP.
 \item An erosion model that can simulate long-term soil erosion
continuously---a continuous simulation model %*** REF
\end{enumerate}

\subsection{Limitations}
\label{sec:LimitationsStage2}
Long term trends of rainfall intensity at the study site were only available at
daily scale. These daily trends cannot be used with the erosion models directly.
Sub-daily rainfall data are required for modelling erosion.
%Thus, rainfall intensity used for the estimation of future soil erosion may
%be considerably different from the actual rainfall intensity changes in the
%future.
Slope data used for erosion simulations are obtained from one site (i.e.\
Woodingdean, South Downs, UK). Thus, there is a chance that the erosion models
may give different results when other slope data are used.
%However, when similar input conditions such as soil type and slope
%characteristics were used, the similar results observed in the present
%research can be expected.

No comparisons of estimated runoff and erosion rates against measured runoff and
erosion rates were possible because of the absence of observed data for the
individual events considered. However, the changes in runoff and erosion rates
from laboratory experiments with similar storm intensity conditions show similar
runoff and erosional responses to the intensity changes. Proper comparisons of
soil erosion can only be made when there are observed erosion rates, which then
can be used as a reference. The validity of the results thus need to be examined
using a different approach.

\section{Past and Present Rates of Rainfall Intensity}
\label{sec:PastAndPresentRatesOfRainfallIntensity}
\subsection{Findings}
\label{sec:FindingsStage3}
No significant trend of rainfall intensity have been found at the research site
with the analysis of multi-scaled rainfall datasets. However, decreasing
tendencies of rainfall amounts in July and March were observed from some
stations. Also, declining number of wet-days per year were noted. With little or
no change in rainfall amounts, these imply subtle increases in daily rainfall
intensity may be expected.
% say more. To what extent are these findings consistent with predictions
% of future rainfall intensity?
Availability of long-term high-resolution rainfall records is paramount for
investigations into trends in rainfall intensity at an appropriate scale for
erosion modelling. Future scenarios of intensity change were constructed for use
in the erosion simulations (Chapter \ref{sec:ESTIMATIONSOFFUTURESOILEROSION}).

\subsection{Limitations}
\label{sec:LimitationsStage3}
Rainfall trends from the observed rainfall data are site- and time-specific. The
other data such as soil and slope data used for estimations were obtained from
one area (Figure \ref{fig:WoodingdeanSite}), and collected for a certain time
period. Although long-term (i.e.\ 10--99 years) observed rainfall data were used
for the intensity trend investigations, high-resolution data---which are
required for intensity analysis and the erosion simulation---were available only
for relatively short periods (i.e.\ 2--13 years) with missing data periods (See
Table \ref{tab:PrecipitationDataUsedForCurrentRainfallTrendInvestigation} and
\ref{tab:DetailsOfDataStations}).

\section{Can we predict erosion rates under future changed rainfall
intensities?}
\label{sec:CanWePredictErosionRatesUnderFutureChangedRainfallIntensities}
%***This section needs to be rewritten.
\subsection{Findings}
\label{sec:FindingsStage4}

%*** REWRITE
The WEPP simulation results suggest that, where mean maximum 30-min peak
intensity of the wet months increases, runoff and erosion increase. Particularly
the amount of erosion increases at a even greater rate than the amount of
runoff: i.e.\ erosion is more sensitive to increased rainfall intensity, compared
with runoff.
%The ratio of erosion increases to the rainfall intensity increase is on the
%order of 2.


\subsection{Limitations}
\label{sec:LimitationsStage4}
This research is mainly based on computerised model simulations. Thus, the
results from the research should not be confused with real observation. The
models merely try to simulate the real soil erosion, based on the known
statistical relationships between processes involved in soil erosion.

\section{Suggestions for Future Research}
\label{sec:SuggestionsForFutureResearch}

The conclusion of this research does not suggest the dismissal of the use of
erosion models for the study on future erosion, but instead it points out that
there are some crucial aspects, which have been identified during the course of
the research, and that they need to be satisfied in order to heighten our
ability to estimate future erosion.

The reatest realisation this research has made is that there are still lots of
questions waiting to be answered in the area of modelling erosion by water,
particularly in relation to future rainfall intensity.

A number of future research topics can be suggested.
\begin{enumerate}
  \item Experiments/measurements of duration/frequency of Within-Storm Gaps
(WSGs) and their implications for soil erosion
  \item Experiments/measurements of Within-Storm Intensity Patterns (WSIPs) and
their implications for soil erosion
  \item Experiments/measurements of Within-Storm Intensity Variations (WSIVs)
and their implications for soil erosion
  \item Developments of erosion models that can make appropriate use of the
information on WSIPs, WSIVs and WSGs
  \item Investigations/predictions of the future trends of WSIPs, WSIVs and WSGs
\end{enumerate}

%\nolinenumbers

%This research has taken three previously set stages: \textsl{Observed
%Precipitation of the Research Site}, \textsl{Rainfall Intensity and
%Erosion} and \textsl{Implications of Climate Change for Future Erosion}.
%All the research questions stated above were dealt as a chapter or
%chapters. The findings and pragmatic shortcomings from the investigation by
%this research were summarised below.

%\subsection{Stage 1: Observed Precipitation of the Research Site}
%\label{sec:Stage1ObservedPrecipitationOfTheResearchSite}

%\paragraph{Findings}
%\label{sec:FindingsStage1}
%Despite the availability of multi-scaled rainfall datasets, no significant
%trend of rainfall intensity can be determined. Although it was not
%conclusive, decreasing tendency of rainfall amount in July and March was
%observed at some stations. Also, because of declining number of annual
%wet-days, subtle increases in daily rainfall intensity may be expected.
%Availability of long-term high-resolution rainfall record is paramount for
%the investigation into the trend in rainfall \emph{intensity}. Controlled
%future scenarios of rainfall intensity change had to be constructed for the
%simulations at the later stage.
%
%\paragraph{Limitations}
%\label{sec:LimitationsStage1}
%Rainfall trends from the observed rainfall date are site- and
%time-specific. The other data used for the model estimation are obtained
%from one area (i.e.\ South Downs, UK) and collected for a certain time
%period. Although long-term observed rainfall data were used for the
%intensity trend analysis, high-resolution data, which are required for
%intensity analysis and the erosion simulation, were only available for
%relatively short periods---and, even with missing data.

%\subsection{Stage 2: Rainfall Intensity and Erosion}
%\label{sec:Stage2RainfallIntensityChangesAndSoilErosion}

%\paragraph{Findings}
%\label{sec:FindingsStage2}
%The improvement of CLIGEN has clear implications for erosion simulations.
%Updated CLIGEN is more sensitive to the changes in rainfall intensity
%(i.e.\ MX~.5P) than the old one. This sensitivity change affects erosion
%estimations more considerably for the regions where low intensity rainfall
%events are dominant in comparison to the regions where high intensity
%rainfall events are dominant.
%
%Rainfall data with higher temporal resolution led to increased erosion
%estimations. Using rainfall data with an inconsistent temporal resolution
%for erosion simulations may result in erroneous erosion estimations.
%Rainfall data scales higher than 15-min may lead to a greater increase in
%runoff and soil loss than that of longer scales (***This statement need to
%be changed).
%
%Intra-storm intensity pattern had an effect on estimations of soil loss.
%Runoff was less affected by intensity patterns than soil loss. A rainfall
%event with a constant intensity produced dramatically less erosion than
%other intensity patterns such as increasing or decreasing. Constant
%intensity is not a good representation of rainfall intensity for an
%rainfall event that is to be used for erosion simulations.
%
%Discontinuous rainfall was expected to produce less runoff and erosion than
%continuous rainfall that has the same rainfall amount because of the
%relatively longer rainfall duration of discontinuous rainfall. However,
%unexpected results, that is overestimations of erosion, from WEPP
%simulations was found. This revealed a design flaw in the rainfall data
%description used in WEPP. WEPP modifies original intensity information of
%breakpoint data by reconstructing the data to ``WEPP-interpreted''
%breakpoint data, and simulates erosion.
%
%To predict future soil erosion that has been affected by future rainfall
%intensity changes, the following are required:
%\begin{enumerate*}
% \item Long-term rainfall data with adequately high temporal resolution
% \item Information about intra-storm intensity patterns
% \item duration and frequency of no-rain periods with a storm
% \item An erosion model that can make a proper use of above rainfall data
%and information (e.g. using data with up to 1440 breakpoints, at least)
% \item An erosion model that can simulate long-term soil erosion
%continuously.
%\end{enumerate*}
%
%\paragraph{Limitations}
%\label{sec:LimitationsStage2}
%Long term trends of rainfall intensity at the study site were only
%available in daily scale. These daily trends can not be used with the
%erosion models directly. Sub-daily rainfall data are required for modelling
%erosion. %Thus, rainfall intensity used for the estimation of future soil
%erosion may be considerably different from the actual rainfall intensity
%changes in the future.
%Slope data used for erosion simulations are obtained from one site (i.e.
%Woodingdean, South Downs, UK). Thus, there is a chance that the erosion
%models may give different results when other slope data are used. %However,
%when similar input conditions such as soil type and slope characteristics
%were used, the similar results observed in the present research can be
%expected.
%
%No direct comparisons of runoff and erosion rates to measured rates were
%possible because of the absence of observed data for the individual events.
%However, the changes in runoff and erosion rates from laboratory
%experiments with similar storm intensity conditions show similar runoff and
%erosional responses to the intensity changes. Proper comparisons of soil
%erosion can only be made when there is observed erosion rates, which then
%can be used as a reference. The validity of the results thus need to be
%supported by a different approach.

%$\Downarrow\Downarrow\Downarrow$ \textbf{still being changed}
%$\Downarrow\Downarrow\Downarrow$

%\subsection{Stage 3: Implications of Climate Change for Future Erosion}
%\label{sec:Stage3ImplicationsOfClimateChangeForFutureSoilErosion}

%\paragraph{Findings}
%\label{sec:FindingsStage3}
%The WEPP simulation results suggest that, where mean maximum 30-min peak
%intensity of the wet months increases, runoff and erosion increase.
%Particularly the amount of erosion increases at a even greater rate than
%the amount of runoff. The ratio of erosion increases to the rainfall
%intensity increase is on the order of 2 (***This statement need to be
%changed).
%
%\paragraph{Limitations}
%\label{sec:LimitationsStage3}
%This research is mainly based on computerised model simulations. Thus, the
%results from the research should not be confused with real observation. The
%models merely try to simulate the real soil erosion, based on the known
%statistical relationships between processes involved in soil erosion.


%%%%%%%%% old version %%%%%

%This research has found that the trend of rainfall at the research site,
%the rainfall data requirement for soil erosion modelling, the effect of
%rainfall intensity patterns on soil erosion, the effect of continuous and
%discontinuous rainfall on soil erosion and the effect of future rainfall
%intensity changes on soil erosion.

%The detailed findings of this thesis are listed by chapter.

%\paragraph{Chapter \ref{sec:RainfallCharacteristicsOfTheStudyArea}}
%\label{sec:ChapterRefSecRainfallCharacteristicsOfTheStudyArea}
%Daily intensity has an increasing trend.
%No significant trend of rainfall intensity can be determined from event
%data because of the short duration of data.
%No significant trend of rainfall intensity can be determined from observed
%rainfall data. %because of the short duration of data.
%Future scenarios were constructed for the later stage.

%\paragraph{Chapter \ref{sec:IMPLICATIONSOFIMPROVEDCLIGEN}}
%\label{sec:ChapterRefSecIMPLICATIONSOFIMPROVEDCLIGEN}
%The improvement in CLIGEN has clear implications for runoff and soil loss
%rate simulations. CLIGEN (v5.2) is sensitive to the changes in rainfall
%intensity, which is parametrized as MX~.5P values in CLIGEN input files.
%The effect of the improvement of CLIGEN is more considerable for the
%regions where low intensity rainfall events are dominant in comparison to
%the regions where high intensity rainfall events are dominant.

%\paragraph{Chapter \ref{sec:EFFECTSOFTEMPORALSCALESOFSTROMDATA}}
%\label{sec:ChapterRefSecEFFECTSOFTEMPORALSCALESOFSTROMDATA}
%Temporally more detailed rainfall data lead to the increased estimation of
%runoff and soil loss rates.
%Rainfall data scales higher than 15-min may lead to a greater increase in
%runoff and soil loss than that of longer scales.

%\paragraph{Chapter
%\ref{sec:EFFECTSOFRAINFALLINTENSITYCHANGESONSOILEROSION}}
%\label{sec:ChapterRefSecEFFECTSOFRAINFALLINTENSITYCHANGESONSOILEROSION}
%Intra-storm intensity pattern affects soil erosion amount. Runoff is not
%much affected. Events with constant low intensity produced dramatically
%less erosion. Constant rainfall intensity is not a good representation of
%rainfall intensity when used in soil erosion simulation.

%\paragraph{Chapter \ref{sec:EFFECTSOFCONTINOUSANDDISCONTINUSSTORM}}
%\label{sec:ChapterRefSecEFFECTSOFCONTINOUSANDDISCONTINUSSTORM}
%Discontinuous rainfall is expected to produce less runoff and erosion than
%continuous rainfall. Model simulation revealed issues with rainfall data
%used in erosion models. WEPP modifies original rainfall intensity data and
%simulates results opposite to those expected.

%\paragraph{Chapter \ref{sec:SUMMARYANDLIMITATIONSOFEROSIONMODELS}}
%\label{sec:ChapterRefSecSUMMARYANDLIMITATIONSOFEROSIONMODELS}
%To predict future soil erosion that has been affected by future rainfall
%intensity changes, the following are required:
%\begin{enumerate}
% \item Long-term rainfall data with 30-min or higher resolution
% \item Intra-storm intensity patterns
% \item The duration and frequency of no-rain periods
% \item An erosion model that can make a proper use of above rainfall data
%and information (e.g. using up to 1440 breakpoints, at least)
% \item An erosion model that can simulate long-term soil erosion
%continuously.
%\end{enumerate}


%\paragraph{Chapter \ref{sec:ESTIMATIONSOFFUTURESOILEROSION}}
%\label{sec:ChapterRefSecESTIMATIONSOFFUTURESOILEROSION}
%The WEPP simulation results suggest that, where mean maximum 30-min peak
%intensity of the wet months increases, runoff and erosion increase.
%Particularly the amount of erosion increases at a even greater rate than
%the amount of runoff. The ratio of erosion increases to the rainfall
%intensity increase is on the order of 2.

%My contributions are the findings of the effects of rainfall intensity
%patterns and inter-storm periods within a storm, so make them noticeable.
%Also, the method for future extreme rainfall intensity changes and trend of
%rainfall intensity of the study site.

%\section{Limitations of This Thesis}
%\label{sec:LimitationOfThisResearch}
%The following limitations of this research are identified.

%There is no observed runoff and soil loss available to compare with the
%simulations---no measurements of runoff or erosion for the each rainfall
%event used.
%
%One site dataset from UK.

%\begin{itemize}
% \item Rainfall trends found by the research are only applicable for the
%research site during the studied period. Data sets used in the research are
%obtained from one area, South Downs, UK. %Yes, but that's what you were ...
%to do
% Although long-term observed data were used for the trend analysis, high
%resolution data which are required for the erosion simulation were only
%available for the short periods. Rainfall intensity trends from the high
%resolution data is only applicable for the studied periods.
% \item Long term rainfall intensity trends at the study site were only
%available at a daily scale and these can not be used with the erosion
%models in this research. Sub-daily rainfall data are required for the
%models. Thus, rainfall intensity used for the estimation of future soil
%erosion may be considerably different from the actual rainfall intensity
%changes in the future.
% \item Slope data used for the erosion simulations are obtained from one
%site (i.e. Woodingdean, South Downs, UK). Thus, erosion models may not give
%the similar outputs if other slope data were used. However, when similar
%input conditions such as soil type and slope characteristics were used, the
%similar results observed in the present research can be expected.
% \item No direct comparisons of runoff and erosion rates to measured rates
%were possible because of the absence of observed data for the individual
%events. However, the changes in runoff and erosion rates from laboratory
%experiments with similar storm intensity conditions show similar runoff and
%erosional responses to the intensity changes. Proper comparisons of soil
%erosion can only be made when there is observed erosion rates, which then
%can be used as a reference. The validity of the results thus need to be
%supported by a different approach.
% \item This research is mainly based on computerised model simulations.
%Thus, the results from the research should not be confused with real
%observation. The models merely try to simulate the real soil erosion, based
%on the known statistical relationships between processes involved in soil
%erosion.
%\end{itemize}
