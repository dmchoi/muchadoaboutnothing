%\linenumbers*
\chapter{CONCLUSION}
\label{sec:CONCLUSION}

The main objective of this research has been to investigate the implications of
change in future rainfall intensity for future soil erosion by water, by
analysing the response of erosion models to arbitrary rainfall intensity
changes, and implicitly, the process understanding on which the the erosion
models are based. Therefore, this research can be regarded as a step towards the
ultimate goal of predicting future soil erosion rates caused by future rainfall
intensity changes.

The current research has been carried out in two parts: \textit{Rainfall
Intensity and Erosion: Model Descriptions and Responses} and
\textit{Implications for Model-based Studies of Future Climate Change and Soil
Erosion}.

In Part I, \textit{Rainfall Intensity and Erosion: Model Descriptions and
Responses}, WEPP, EUROSEM and RillGrow were used to investigate the:
\begin{itemize}
  \item information regarding rainfall intensity which is needed to simulate
soil erosion,
  \item responses of selected erosion models to various changes in rainfall
intensity,
  \item representation of the effects of rainfall intensity on erosion within
each erosion model,
  \item applicability of these erosion models for future research.
\end{itemize}

Then, in Part II, \textit{Implications for Model-based Studies of Future Climate
Change and Soil Erosion}, the best approaches for building scenarios of future
rainfall intensity changes were discussed based on the findings from the
previous part. This was then followed by the erosion simulations by WEPP
with this future intensity scenario. Part II thus aimed to:
\begin{itemize}
  \item suggest the best currently-available way of investigating impacts of
rainfall intensity changes on future erosion and
  \item test this method using scenarios of future change in rainfall intensity.
\end{itemize}

To achieve these aims of this research, following research questions was
tackled:
\begin{enumerate}[{Question} 1.]
  \item What role does rainfall intensity play in the process descriptions which
comprise erosion models?
  \item Assuming that we use a model to predict erosion rates under a future
climate, and that this future climate has different rainfall intensity from the
present, what information (both regarding climate, and regarding erosion
processes) do we need to make predictions of soil erosion rates under that
future climate?
  \item From the above, are we in a position to predict soil erosion under the
future climate? If not, what more is necessary?
\end{enumerate}

In this chapter, the main findings of each part of this thesis are summarised
with respect to the above research questions. The chapter consists of three
sections. The first section outlines the findings from Part I. In the
second section, these findings from Part I are linked with the research
questions and answers are discussed. Finally, the third section outlines
perspectives for future research on effects of rainfall intensity changes on
soil erosion.

\section{Summary of Model Responses to Rainfall Intensity}
\label{sec:SummaryofModelResponsestoRainfallIntensityChange}

\subsection{Effect of Temporal Resolution of Storm Data}
There is a close relationship between temporal resolution of rainfall data
and soil erosion estimations. When temporal resolutions of rainfall data
changes (increase or decrease), rainfall intensities also changes (decrease or
increase). Temporal resolution of rainfall data thus is closely related to WSIVs
as well. Model simulation results can be up to about 30 times different from
the ``original results'' if rainfall data with low temporal resolution are used.
This magnitude is so great that it might mean from almost no erosion to
disastrous events.

In terms of soil loss, WEPP is more sensitive to the change of temporal
data resolutions than EUROSEM. Breakpoint data are preferred to CLIGEN data
for erosion modelling because they are less affected by the change of temporal
resolutions than CLIGEN. Where it is available, 15-min breakpoint rainfall data
are recommended for erosion simulations.

\subsection{Effect of Within-Storm Gap}
It is not evident if WSG has positive or negative effects on soil erosion
estimations. However, removing WSG from original rainfall data affects the
model simulation results by WEPP and EUROSEM. Therefore, it is not recommended
to remove WSG from the original rainfall data. It is a best practice to
use rainfall data which have all necessary information for erosion modelling.

When breakpoint data with a temporal resolution shorter than 60-min temporal
resolution is used, WEPP modifies original rainfall intensity information and
simulates erroneous erosion rates. This is a major problem for current
research and for our ability to predict erosion. This is also a major model
fault for WEPP. Even if 15-min breakpoint rainfall data, as suggested
previously, are available for WEPP simulations, WEPP is not capable of
utilising original rainfall intensity information and simulates unrealistic soil
erosion. Until this problem is resolved, interpreting WEPP outputs which are
generated with breakpoint data requires additional cautions. Therefore, an
urgent improvement work is needed for WEPP.

\subsection{Effect of Within-Storm Intensity Pattern}
WSIP also has effects on soil loss estimations. Effects of WSIP on runoff
estimations are however small implying that WSIP affects estimations of soil
loss rates more. WSIP of a rainfall storm with high intensity has less impacts
on soil loss estimations than WSIP of a rainfall storm with low intensity.
Despite varying responses of WEPP, EUROSEM and RillGrow to WSIP, constant WSIP
produced the least soil loss rates for WEPP and RillGrow. There are urgent needs
for more laboratory or field experiments that can be compared with the model
results in order to improve model predictabilities.

\section{Answers to Research Questions}
\label{sec:ResearchQuestionsLinkedtoTheResults}

\subsection{What role does rainfall intensity play in the process
descriptions which comprise erosion models?}
\label{sec:RainfallIntensityAndProcessDescriptions}
There is a difference between the theoretical treatment of rainfall intensity in
erosion models, and the practical handling of it. For example, increased
rainfall intensity is not always accompanied by estimations of increased runoff
and soil loss as shown in Part I of this thesis. Effects of rainfall intensity
changes on erosion modelling largely depend on the process descriptions on which
erosion models are based. However, rainfall intensity undoubtedly plays an
important role in estimating runoff and soil loss rates.

As shown previously, Within-Storm Intensity Variation (WSIV) are related to
temporal resolution of rainfall data, Within-Storm Gap (WSG) and Within-Storm
Intensity Pattern (WSIP). Temporal resolution of rainfall data affects the
results of erosion simulations. When temporal resolution of rainfall data
increases, the amount of estimated soil loss also increases. Thus, simulation
studies which use rainfall data with different temporal resolutions will give
different amounts of estimated soil loss.

Secondly, WSG also have a marked effect on erosion estimations. Discontinuous
storms (i.e.\ with WSGs) would be expected to produce less runoff and soil loss
than an equivalent continuous storm (i.e.\ a storm with the same total rainfall,
but with WSGs removed), since the longer duration of the discontinuous storm
reduces the average intensity for the storm, compared with the continuous storm.
However, an unexpected result was found from WEPP simulations that more soil
loss was estimated with discontinuous rainfall than with continuous rainfall.
Original rainfall data is internally reconstructed to ``WEPP-modified'' data,
and this modified data is used in the simulations.

Lastly, WSIP affects the results of erosion simulations. Rainfall events with a
constant WSIP produced notably small erosion rates compared to those with
other WSIPs. WSIP affects estimations of soil loss rates more than estimations
of runoff.

\paragraph{Limitation of model simulation results}
\label{sec:LimitationsStage1}
No comparisons of estimated runoff and soil loss rates against measured runoff
and soil loss rates were possible because of the lack of observed erosion data
for the individual events considered. However, the changes in runoff and erosion
rates from laboratory experiments with similar storm intensity patterns show
similar runoff and erosional responses to the change of WSIP. Better comparisons
between modelling results and observed data can only be made when there is
more observational data.

This research is mainly based on computerised model simulations. Thus, the
results from the research should not be confused with real observation. The
models merely try to simulate the real soil erosion, based on the known
statistical relationships between processes involved in soil erosion.

\subsection{What information do we need to make predictions of soil erosion
rates under future climate?}
\label{sec:WeNeedToPredictErosionUnderFutureChangedRainfallIntensities}
To predict soil erosion rates under future climate, the following is required:
\begin{enumerate}
  \item Long-term rainfall data, preferably breakpoint data, with adequately
high temporal resolution---high enough to capture details of Within-Storm
Intensity Variations (WSIVs)
  \item Quantity (\%) of Within-Storm Gaps (WSGs)
%*** how detailed? i.e. minimum size of no-rain gap which must be known?
  \item Information about Within-Storm Intensity Patterns (WSIPs)
  \item Erosion model that can make proper use of rainfall intensity
information such as WSIV, WSG and WSIP
  \item Erosion model that can simulate long-term soil erosion continuously---a
continuous simulation model.
\end{enumerate}
%The improvement of CLIGEN has clear implications for erosion simulations.
%Updated CLIGEN is more sensitive to the changes in rainfall intensity (i.e.\
%MX~.5P) than the old one. This sensitivity change affects erosion estimations
%more considerably for the regions where low intensity rainfall events are
%dominant in comparison to the regions where high intensity rainfall events are
%dominant.
%Temporal resolution of rainfall data affect results of erosion simulations. As
%temporal resolutions increase, amounts of erosion estimations increase. Using
%rainfall data with an inconsistent temporal resolution for erosion simulations
%may result in erroneous erosion estimations. Higher temporal resolution is
%needed when rainfall data that have high WSIV (Within-Storm Intensity
%Variation) are used for erosion estimations.
%higher than 15-min may lead to a greater increase in runoff and soil loss than
%that of longer scales (***This statement need to be changed).

%Within-Storm Gap (WSGs) have an effect on erosion estimations. Discontinuous
%rainfall events (i.e.\ WSGs included) were expected to produce less runoff and
%soil loss than continuous rainfall events (i.e.\ WSGs removed) that has the
%same rainfall amount as discontinuous events. This is because relatively longer
%rainfall durations of discontinuous rainfall events imply that the events have
%relatively less intense intensity than continuous events. However, unexpected
%results from WEPP simulations found that WEPP estimates more soil loss with
%discontinuous rainfall than with continuous rainfall. This revealed a design
%flaw in the rainfall data description used in WEPP. WEPP modifies original
%intensity information of breakpoint data by reconstructing the data to
%``WEPP-modified'' data, and uses this ``reconstructed'' data for simulations
%of erosion.

%Within-Storm Intensity Pattern (WSIP) have an effect on estimations of soil
%erosion. WSIPs have more effects on soil loss than runoff. Rainfall events with
%a constant intensity produced dramatically less erosion amounts than other
%WSIPs such as increasing or decreasing. Constant intensity is not a good
%representation of rainfall intensity for a natural rainfall event as far as
%erosion modelling is concerned.

%\paragraph{Difficulties involved obtaining recommended rainfall data}
%\label{sec:LimitationsStage2}
%Long term trends of rainfall intensity at the study site were only available at
%daily resolution. These daily trends cannot be used with the erosion models
%directly. Sub-daily rainfall data are required for modelling erosion.
%Thus, rainfall intensity used for the estimation of future soil erosion may
%be considerably different from the actual rainfall intensity changes in the
%future.
%Slope data used for erosion simulations are obtained from one site (i.e.\
%Woodingdean, South Downs, UK). Thus, there is a chance that the erosion models
%may give different results when other slope data are used.
%However, when similar input conditions such as soil type and slope
%characteristics were used, the similar results observed in the present
%research can be expected.

\subsection{Are we in a position to predict soil erosion under the future
climate? If not, what more is necessary?}
\label{sec:PastAndPresentRatesOfRainfallIntensity}
Based on the findings from this research, the simplest answer to this question
would be ``No, we are not.'' However, this does not mean to say that it is
impossible forever. We just are not \emph{YET} in a position to predict soil
erosion under the future climate with reasonably acceptable uncertainties.

First reason is that there are some major disagreements found between rainfall
data availability and the data requirement suggested by this research. For
example, in order to determine effects of rainfall intensity changes on soil
erosion, we need high resolution rainfall data such as tipping-bucket recorded
data. However, the availability of such data is relatively low because of, for
example, data storage issues as discussed previously. The resolution of rainfall
data need to be high enough to hold adequate rainfall intensity information such
as WSIV, WSG, WSIP for erosion models.

Secondly, erosion models still have inadequate process descriptions for rainfall
intensity. For example, WEPP modifies the accumulated time of original
breakpoint data and changes rainfall intensity of original storm before it uses
them for erosion simulations. Also, WEPP and EUROSEM have the limits on the
total number of breakpoint they can process.

Despite these problems, this research does not suggest the dismissal of the use
of erosion models for the study on future erosion, but instead it points out
that there are some important aspects, which have been identified during the
course of the research, and that they need to be satisfied in order to heighten
our ability to estimate future erosion.
Therefore, this research assists in improving the performance of erosion
models with respect to changes of rainfall intensity by highlighting where
current problem exists.

In conclusion, greater knowledge found here will, once future changes in
rainfall intensity become better known and appropriate rainfall data become
available, improve our ability to estimate future rates of erosion.

\section{Perspectives for Future Research}
\label{sec:SuggestionsForFutureResearch}

The greatest realisation this research has made is that there are still lots of
questions waiting to be answered in the area of modelling erosion by water,
particularly in relation to future rainfall intensity.

Therefore, a number of future research topics can be suggested.
\begin{enumerate}
  \item Laboratory or field experiments on effects of Within-Storm Gaps (WSGs)
  \item More laboratory experiments on effects of Within-Storm Intensity
Patterns (WSIPs) with more complex intensity patterns
  \item Further investigation of Within-Storm Intensity Variations (WSIVs):
determination and quantification of WSIV
  \item Developments/improvements of erosion models that can make appropriate
use of the rainfall information (WSIPs, WSIVs and WSGs)
  \item Investigations and predictions of the trends of WSIPs, WSIVs and WSGs
\end{enumerate}

%\nolinenumbers

%This research has taken three previously set stages: \textsl{Observed
%Precipitation of the Research Site}, \textsl{Rainfall Intensity and
%Erosion} and \textsl{Implications of Climate Change for Future Erosion}.
%All the research questions stated above were dealt as a chapter or
%chapters. The findings and pragmatic shortcomings from the investigation by
%this research were summarised below.

%\subsection{Stage 1: Observed Precipitation of the Research Site}
%\label{sec:Stage1ObservedPrecipitationOfTheResearchSite}

%\paragraph{Findings}
%\label{sec:FindingsStage1}
%Despite the availability of multi-scaled rainfall datasets, no significant
%trend of rainfall intensity can be determined. Although it was not
%conclusive, decreasing tendency of rainfall amount in July and March was
%observed at some stations. Also, because of declining number of annual
%wet-days, subtle increases in daily rainfall intensity may be expected.
%Availability of long-term high-resolution rainfall record is paramount for
%the investigation into the trend in rainfall \emph{intensity}. Controlled
%future scenarios of rainfall intensity change had to be constructed for the
%simulations at the later stage.
%
%\paragraph{Limitations}
%\label{sec:LimitationsStage1}
%Rainfall trends from the observed rainfall date are site- and
%time-specific. The other data used for the model estimation are obtained
%from one area (i.e.\ South Downs, UK) and collected for a certain time
%period. Although long-term observed rainfall data were used for the
%intensity trend analysis, high-resolution data, which are required for
%intensity analysis and the erosion simulation, were only available for
%relatively short periods---and, even with missing data.

%\subsection{Stage 2: Rainfall Intensity and Erosion}
%\label{sec:Stage2RainfallIntensityChangesAndSoilErosion}

%\paragraph{Findings}
%\label{sec:FindingsStage2}
%The improvement of CLIGEN has clear implications for erosion simulations.
%Updated CLIGEN is more sensitive to the changes in rainfall intensity
%(i.e.\ MX~.5P) than the old one. This sensitivity change affects erosion
%estimations more considerably for the regions where low intensity rainfall
%events are dominant in comparison to the regions where high intensity
%rainfall events are dominant.
%
%Rainfall data with higher temporal resolution led to increased erosion
%estimations. Using rainfall data with an inconsistent temporal resolution
%for erosion simulations may result in erroneous erosion estimations.
%Rainfall data scales higher than 15-min may lead to a greater increase in
%runoff and soil loss than that of longer scales (***This statement need to
%be changed).
%
%Intra-storm intensity pattern had an effect on estimations of soil loss.
%Runoff was less affected by intensity patterns than soil loss. A rainfall
%event with a constant intensity produced dramatically less erosion than
%other intensity patterns such as increasing or decreasing. Constant
%intensity is not a good representation of rainfall intensity for an
%rainfall event that is to be used for erosion simulations.
%
%Discontinuous rainfall was expected to produce less runoff and erosion than
%continuous rainfall that has the same rainfall amount because of the
%relatively longer rainfall duration of discontinuous rainfall. However,
%unexpected results, that is overestimations of erosion, from WEPP
%simulations was found. This revealed a design flaw in the rainfall data
%description used in WEPP. WEPP modifies original intensity information of
%breakpoint data by reconstructing the data to ``WEPP-interpreted''
%breakpoint data, and simulates erosion.
%
%To predict future soil erosion that has been affected by future rainfall
%intensity changes, the following are required:
%\begin{enumerate*}
% \item Long-term rainfall data with adequately high temporal resolution
% \item Information about intra-storm intensity patterns
% \item duration and frequency of no-rain periods with a storm
% \item An erosion model that can make a proper use of above rainfall data
%and information (e.g. using data with up to 1440 breakpoints, at least)
% \item An erosion model that can simulate long-term soil erosion
%continuously.
%\end{enumerate*}
%
%\paragraph{Limitations}
%\label{sec:LimitationsStage2}
%Long term trends of rainfall intensity at the study site were only
%available in daily scale. These daily trends can not be used with the
%erosion models directly. Sub-daily rainfall data are required for modelling
%erosion. %Thus, rainfall intensity used for the estimation of future soil
%erosion may be considerably different from the actual rainfall intensity
%changes in the future.
%Slope data used for erosion simulations are obtained from one site (i.e.
%Woodingdean, South Downs, UK). Thus, there is a chance that the erosion
%models may give different results when other slope data are used. %However,
%when similar input conditions such as soil type and slope characteristics
%were used, the similar results observed in the present research can be
%expected.
%
%No direct comparisons of runoff and erosion rates to measured rates were
%possible because of the absence of observed data for the individual events.
%However, the changes in runoff and erosion rates from laboratory
%experiments with similar storm intensity conditions show similar runoff and
%erosional responses to the intensity changes. Proper comparisons of soil
%erosion can only be made when there is observed erosion rates, which then
%can be used as a reference. The validity of the results thus need to be
%supported by a different approach.

%$\Downarrow\Downarrow\Downarrow$ \textbf{still being changed}
%$\Downarrow\Downarrow\Downarrow$

%\subsection{Stage 3: Implications of Climate Change for Future Erosion}
%\label{sec:Stage3ImplicationsOfClimateChangeForFutureSoilErosion}

%\paragraph{Findings}
%\label{sec:FindingsStage3}
%The WEPP simulation results suggest that, where mean maximum 30-min peak
%intensity of the wet months increases, runoff and erosion increase.
%Particularly the amount of erosion increases at a even greater rate than
%the amount of runoff. The ratio of erosion increases to the rainfall
%intensity increase is on the order of 2 (***This statement need to be
%changed).
%
%\paragraph{Limitations}
%\label{sec:LimitationsStage3}
%This research is mainly based on computerised model simulations. Thus, the
%results from the research should not be confused with real observation. The
%models merely try to simulate the real soil erosion, based on the known
%statistical relationships between processes involved in soil erosion.


%%%%%%%%% old version %%%%%

%This research has found that the trend of rainfall at the research site,
%the rainfall data requirement for soil erosion modelling, the effect of
%rainfall intensity patterns on soil erosion, the effect of continuous and
%discontinuous rainfall on soil erosion and the effect of future rainfall
%intensity changes on soil erosion.

%The detailed findings of this thesis are listed by chapter.

%\paragraph{Chapter \ref{sec:RainfallCharacteristicsOfTheStudyArea}}
%\label{sec:ChapterRefSecRainfallCharacteristicsOfTheStudyArea}
%Daily intensity has an increasing trend.
%No significant trend of rainfall intensity can be determined from event
%data because of the short duration of data.
%No significant trend of rainfall intensity can be determined from observed
%rainfall data. %because of the short duration of data.
%Future scenarios were constructed for the later stage.

%\paragraph{Chapter \ref{sec:IMPLICATIONSOFIMPROVEDCLIGEN}}
%\label{sec:ChapterRefSecIMPLICATIONSOFIMPROVEDCLIGEN}
%The improvement in CLIGEN has clear implications for runoff and soil loss
%rate simulations. CLIGEN (v5.2) is sensitive to the changes in rainfall
%intensity, which is parametrized as MX~.5P values in CLIGEN input files.
%The effect of the improvement of CLIGEN is more considerable for the
%regions where low intensity rainfall events are dominant in comparison to
%the regions where high intensity rainfall events are dominant.

%\paragraph{Chapter \ref{sec:EFFECTSOFTEMPORALSCALESOFSTROMDATA}}
%\label{sec:ChapterRefSecEFFECTSOFTEMPORALSCALESOFSTROMDATA}
%Temporally more detailed rainfall data lead to the increased estimation of
%runoff and soil loss rates.
%Rainfall data scales higher than 15-min may lead to a greater increase in
%runoff and soil loss than that of longer scales.

%\paragraph{Chapter
%\ref{sec:EFFECTSOFRAINFALLINTENSITYCHANGESONSOILEROSION}}
%\label{sec:ChapterRefSecEFFECTSOFRAINFALLINTENSITYCHANGESONSOILEROSION}
%Intra-storm intensity pattern affects soil erosion amount. Runoff is not
%much affected. Events with constant low intensity produced dramatically
%less erosion. Constant rainfall intensity is not a good representation of
%rainfall intensity when used in soil erosion simulation.

%\paragraph{Chapter \ref{sec:EFFECTSOFCONTINOUSANDDISCONTINUSSTORM}}
%\label{sec:ChapterRefSecEFFECTSOFCONTINOUSANDDISCONTINUSSTORM}
%Discontinuous rainfall is expected to produce less runoff and erosion than
%continuous rainfall. Model simulation revealed issues with rainfall data
%used in erosion models. WEPP modifies original rainfall intensity data and
%simulates results opposite to those expected.

%\paragraph{Chapter \ref{sec:SUMMARYANDLIMITATIONSOFEROSIONMODELS}}
%\label{sec:ChapterRefSecSUMMARYANDLIMITATIONSOFEROSIONMODELS}
%To predict future soil erosion that has been affected by future rainfall
%intensity changes, the following are required:
%\begin{enumerate}
% \item Long-term rainfall data with 30-min or higher resolution
% \item Intra-storm intensity patterns
% \item The duration and frequency of no-rain periods
% \item An erosion model that can make a proper use of above rainfall data
%and information (e.g. using up to 1440 breakpoints, at least)
% \item An erosion model that can simulate long-term soil erosion
%continuously.
%\end{enumerate}


%\paragraph{Chapter \ref{sec:ESTIMATIONSOFFUTURESOILEROSION}}
%\label{sec:ChapterRefSecESTIMATIONSOFFUTURESOILEROSION}
%The WEPP simulation results suggest that, where mean maximum 30-min peak
%intensity of the wet months increases, runoff and erosion increase.
%Particularly the amount of erosion increases at a even greater rate than
%the amount of runoff. The ratio of erosion increases to the rainfall
%intensity increase is on the order of 2.

%My contributions are the findings of the effects of rainfall intensity
%patterns and inter-storm periods within a storm, so make them noticeable.
%Also, the method for future extreme rainfall intensity changes and trend of
%rainfall intensity of the study site.

%\section{Limitations of This Thesis}
%\label{sec:LimitationOfThisResearch}
%The following limitations of this research are identified.

%There is no observed runoff and soil loss available to compare with the
%simulations---no measurements of runoff or erosion for the each rainfall
%event used.
%
%One site dataset from UK.

%\begin{itemize}
% \item Rainfall trends found by the research are only applicable for the
%research site during the studied period. Data sets used in the research are
%obtained from one area, South Downs, UK. %Yes, but that's what you were ...
%to do
% Although long-term observed data were used for the trend analysis, high
%resolution data which are required for the erosion simulation were only
%available for the short periods. Rainfall intensity trends from the high
%resolution data is only applicable for the studied periods.
% \item Long term rainfall intensity trends at the study site were only
%available at a daily scale and these can not be used with the erosion
%models in this research. Sub-daily rainfall data are required for the
%models. Thus, rainfall intensity used for the estimation of future soil
%erosion may be considerably different from the actual rainfall intensity
%changes in the future.
% \item Slope data used for the erosion simulations are obtained from one
%site (i.e. Woodingdean, South Downs, UK). Thus, erosion models may not give
%the similar outputs if other slope data were used. However, when similar
%input conditions such as soil type and slope characteristics were used, the
%similar results observed in the present research can be expected.
% \item No direct comparisons of runoff and erosion rates to measured rates
%were possible because of the absence of observed data for the individual
%events. However, the changes in runoff and erosion rates from laboratory
%experiments with similar storm intensity conditions show similar runoff and
%erosional responses to the intensity changes. Proper comparisons of soil
%erosion can only be made when there is observed erosion rates, which then
%can be used as a reference. The validity of the results thus need to be
%supported by a different approach.
% \item This research is mainly based on computerised model simulations.
%Thus, the results from the research should not be confused with real
%observation. The models merely try to simulate the real soil erosion, based
%on the known statistical relationships between processes involved in soil
%erosion.
%\end{itemize}
