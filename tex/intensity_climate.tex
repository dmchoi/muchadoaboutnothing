\section{Rainfall Intensity and Climate Change}
\label{sec:RainfallIntensityAndClimateChange}

Many studies using GCMs predict an increase in global average precipitation in
response to global warming induced by greenhouse gases
\citep{houghton1996-climate,jones2001-1337,ipcc2001-881,ipcc2001-1032,
ipcc2007-impact,ipcc2007-physical}. This increase in global average
precipitation has been based on the assumption that an increasing global-mean
temperature will intensify the hydrological cycle
\citep{nearing2005-131}. The IPCC reported that there has been a very
likely increase in precipitation during the 20th century in the mid-to-high
latitudes of the Northern Hemisphere \citep{ipcc2001-881,ipcc2007-physical}.
Climate models are also predicting a continued increase in intense precipitation
events during the 21st century \citep{ipcc2001-1032,ipcc2007-impact}.

In addition, there has been a number of investigations using observed data
that provided some evidences for a significant increase in extreme precipitation
\citep{karl1995-217,karl1998-231,osborn2000-347,osborn2002-1313}.
\citet{karl1995-217} and \citet{karl1998-231} observed increases in extreme
precipitation (greater than 50 mm per day) in the United States using historical
data over the period 1910--1996. \citet{osborn2000-347} and
\citet{osborn2002-1313} also observed an increasing trend in intense daily
precipitation over the period 1961--2000 in the United Kingdom. They found that,
on average, precipitations were becoming more intense in winter and less intense
in summer.

The findings by \citet{osborn2000-347} and \citet{osborn2002-1313} are generally
consistent with the results from the GCM simulations
\citep{jones1997-265,jones2001-1337}. However,
\citet{ipcc2001-1032,ipcc2001-881} indicated that potential changes in intense
rainfall frequency are difficult to infer from global climate models, largely
because of coarse spatial resolution. The ability of GCM integrations and
operational analyses to simulate realistic precipitation patterns, spatially and
seasonally, is also generally not as good as the ability to predict temperature
\citep{mcguffie1999-1}. The likelihood of finding real trends in the frequency
of extreme events becomes lower the more extreme the event
\citep{frei2001-1568}. The same authors demonstrate this by applying known
trends in the scale parameter to synthetic data series, and then attempt to
identify statistically significant trends in the frequency of various extreme
events.

%******$Summarise what are those physical reasons in \citealp{trenberth2000-12}.

There are various physical reasons (see \citealp{trenberth2000-12}) why a large
increase in the magnitude of heavy precipitation may occur with only a
correspondingly small increase in mean precipitation. It is even possible that
heavy precipitation occurrence could increase when mean precipitation decreases,
if there is a more radical change in the precipitation distribution
\citep{osborn2002-1313}.

A study by \citet{nearing2001-229} estimated potential changes in rainfall
erosivity in the United States during the 21st century under climate change
scenarios. He concluded that, across the United States over an 80 year
period, the magnitude of average changes in rainfall erosivity was 16--58\%.
This variability in the magnitude was due to the method (two GCM models and
two scenarios) that he used to predict the changes in rainfall erosivity.
Regardless of which method was used, he suggested that changes in erosivity will
be critical at certain locations.

In order to run a soil erosion model such as WEPP (Water Erosion Prediction
Project, See Section \ref{sec:WaterErosionPredictionProjectWEPP} for more
details), for example, various weather parameters for each day of the
simulation period are required \citep{flanagan1995-usda}. These weather
variables (e.g. rainfall depth and duration, peak storm intensity and time to
peak, minimum and maximum temperatures, dew point temperature, solar radiation,
wind speed and direction) can either be generated by CLIGEN (CLImate GENerator,
See Section \ref{sec:ClimateGeneratorCLIGEN} for more details) or compiled
manually from observed climate data.

Generating climate data for studies on future soil erosion is not a simple task,
even with today's climate data, as a starting point, since all erosion
predictions must involve modelling extreme weather events. Extreme weather
events (e.g., heavy showers, gusts and tornadoes) are rare and occur on the
synoptic and even smaller temporal and spatial resolutions
\citep{schubert1997-223,katz1999-133,coppus2002-1365}. Long integrations of very
high-resolution models are required to simulate those extreme events and even
then, there is little prospect that sub-synoptic scale events can be
successfully resolved in GCMs. GCM grid sizes are too large to properly capture
convective elements in the atmosphere, so that precipitation within a short
period (e.g., one day) is poorly reproduced by GCMs \citep{schubert1997-223}.

There are a few ways for resolving this scale issues with GCM data. One way is
by using climate data generated directly by Regional Climate Models (RCMs) that
are capable of generating climate data with a sub-daily resolution (i.e.
20-min). Another can be achieved by downscaling. There are several approaches
for downscaling GCM data into regional scale. \citet{wilby1997-530} divided
downscaling into four categories: regression methods, weather pattern
(circulation)-based approaches, stochastic weather generators and limited-area
climate models. Among these approaches, circulation-based downscaling methods
perform well in simulating present observed and model-generated daily
precipitation characteristics, but regression methods are preferred because of
its ease of implementation and low computation requirements. RCM data and
downscaled data allow predictions to be made at a finer resolution than GCMs. All of
these methods are widely accepted methods that were often chosen to generate
climate data with a sub-daily resolution.
%******this is an important point, see Dave's email on 11/10/09 17:40:17. If
%GCMs cannot reproduce present day rain intensity, how can they be expected to
%give us information about future rain intensity? Say more.

Lastly, \citet{ipcc2001-1032} reported generalised results from the analysis
of five regional climate change simulations. Although scenarios for
precipitation produced by these experiments varied widely among models and from
region to region, the results provide very important working envelopes for this
research. The results related to precipitation are summarised as follows:
\begin{enumerate}
  \item Regional precipitation error spanned a wide range, with values as
extreme as approximately $-$90\% or $+$200\%.
  \item Simulated precipitation sensitivity to doubled CO$_2$ was mostly
in the range of $-$20\% to $+$20\% of the control value.
  \item Overall, the precipitation errors were greater than the simulated
changes. It can be expected that, due to relatively high temporal and spatial
variability in precipitation, temperature changes are more likely to be
statistically significant than precipitation changes.
\end{enumerate}
