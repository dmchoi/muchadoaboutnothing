\begin{abstractseparate}
\phantomsection % required for using hyperref
\addcontentsline{toc}{chapter}{Extended Abstract}
%Word limit: 1500
\noindent
%Soil is one of the essentials for life on Earth, together with water, air
%and sunlight. Undoubtedly, soil is an important resource for the survival
%of the human race as most agricultural products are cultivated on soil.
%Sustaining soil conditions is therefore a crucial issue where soil
%nutrients, which control crop yields, fibre and fuel, are scarce.
Globally, soil erosion by water is a serious environmental problem. One of many
approaches to study this problem is to simulate soil erosion using computer
models, which help one to understand complex interactions between various
conditions of land use, soil types and climate.
%
Existing simulation studies of the effects of future climate change upon erosion
indicate that, under land usages that leave the soil unprotected, even minor
increases in rainfall amounts are likely to result in disproportionately large
increases in erosion. Such studies, however, invariably make the simplifying
assumption that distributions of future rainfall intensities remain unchanged
from the present. This is unlikely to be the case. In the latest IPCC
(Intergovernmental Panel on Climate Change) report, global average water vapour
concentration and precipitation are projected to increase during the 21st
century. This implies that there will be an increase in the frequency and
magnitude of heavy rainfall. Future climate change will certainly affect
rainfall intensities, and thus soil erosion, but our ability to forecast future
rainfall intensities is limited by the shortcomings of GCMs (General Circulation
Models).

Therefore, the main objective of this research is to investigate possible
implications of climate change for future erosion with reference to rainfall
intensity changes by analysing the response of erosion models to arbitrary
rainfall intensity changes, and implicitly the process understanding on which
the models are based.
Thus, this research is a step towards the ultimate goal of predicting future
rates of soil erosion caused by future rainfall intensity changes.

Three soil erosion models, WEPP, EUROSEM, and RillGrow are employed to
investigate impacts of various rainfall intensities on runoff and soil loss
rates. Two extreme daily rainfall events in summer and autumn are subjectively
selected from the tipping-bucket rainfall data, and runoff and soil losses are
simulated using three erosion models. Modelling soil erosion requires highly
detailed rainfall-intensity information. Estimated runoff and soil loss rates
with high resolution rainfall data are greater than those with low temporal
resolution rainfall data. WSIPs (Within-Storm Intensity Patterns) affect soil
erosion amount, although runoff was not much affected. An additional daily
rainfall event with WSGs (Within-Storm Gaps) within a storm is also selected to
highlight effects of intra-storm pause within a storm on soil erosion.
For a given depth of rainfall, events with constant low intensity produced
dramatically less erosion: thus it appears that assuming a constant (or
averaged) intensity throughout a storm does not provide a good representation of
natural rainfall with its continuously varying intensity.
Analyses of outputs from WEPP simulations reveal a problem that WEPP modifies
original rainfall intensity and, thus, simulates erroneous runoff and erosion
rates.

Monthly 0.5\textdegree\ grid data for 100 years are analysed to draw outlines of
rainfall trends in the study area. Trends of daily rainfall amount, number of
raindays, simple daily intensity index, number of raindays with rainfall amount
$\geq$10 mm, and number of raindays with rainfall amount $\geq$20 mm are also
investigated with 9--93 years long daily station data. Detailed
rainfall-intensity patterns in the study area are examined using tipping-bucket
rainfall gauge data. Analysis of three observational rainfall datasets show
increases in frequency of extreme events, and an increasing trend in daily
rainfall intensity for the future.
Long-term monthly 0.5\textdegree\ grid data analysis shows a statistically
significant decreasing trend in July rainfall amount over the 1901--2000 period.
With daily observational data, March rainfall amounts in the last two decades
and July rainfall amounts in the last decade show a downward trend although
these are not significant.  Simultaneously, the numbers of raindays per month
show downward trends. Annual daily rainfall-intensity over 1904--1996 has
increased significantly. This is mainly the result of an increased number of
extreme rainfall events ($\geq$10 mm) compared to an annual number of raindays.

Future soil erosion rates are estimated using WEPP and CLIGEN (Climate
Generator). 30 year-long weather is generated by CLIGEN. Likely future rainfall
frequency and intensity are anticipated by changing the mean maximum 30 minutes
peak intensity. No future rainfall amount change is assumed. WEPP simulation
results suggest that where mean maximum 30-min peak intensity of the wet months
increases soil erosion increases at a greater rate than runoff.

Several further investigations are suggested on: measurement of runoff and soil
loss with discontinuous storms and their implications for soil erosion; an
investigation of relationships between duration of no-rain period and soil
erosion; development of an erosion model that can fulfil the requirements
suggested in this thesis; an investigation of the relationship of intra-storm
intensity patterns and data resolution (temporal and spatial); an investigation of
the trend of intra-storm intensity and no-rain periods.
%Storm patterns result in varying simulated runoff and soil loss although
%the simulation results are rather different from measured runoff and soil
%loss. It is evident that storm pattern is an important affecting factor of
%soil erosion modelling.  Inter-storm gaps within a storm also play an
%important role in soil erosion modelling. Discarding inter-storm gaps
%within a storm for soil erosion modelling results in considerable
%overestimation of runoff and soil loss as rainfall duration is effectively
%shortened.
\end{abstractseparate}
