\begin{abstract}
\phantomsection % required for using hyperref
\addcontentsline{toc}{chapter}{\abstractname}
% Word limit: 300
\noindent
%Soil is one of the essentials for life on Earth, together with water, air
%and sunlight. Undoubtedly, soil is an important resource for the survival
%of the human race as most agricultural products are cultivated in soil.
%Sustaining agriculturally optimal soil conditions is therefore a crucial
%issue, particularly where soil nutrients, which control crop yields, fibre
%and fuel, are scarce. Globally, soil erosion by water is a serious
%environmental problem. One of many approaches to study this problem is to
%simulate soil erosion using computer models, which help one to understand
%complex interactions between various conditions of land use, soil types and
%climate.
%
%Previously published simulation studies of the effects of future climate
%change upon erosion indicate that, under land usages that leaves the soil
%unprotected, even minor increases in rainfall amounts are likely to result
%in disproportionately large increases in erosion. Such studies, however,
%invariably make the simplifying assumption that distributions of future
%rainfall intensities remain unchanged from the present. This is unlikely to
%be the case. In the latest IPCC (Intergovernmental Panel on Climate Change)
%report, global average water vapour concentration and precipitation are
%projected to increase during the 21st century. This implies that there will
%be an increase in the frequency and magnitude of heavy rainfall. Future
%climate change will certainly affect rainfall intensities, and thus soil
%erosion, but our ability to forecast how future rainfall intensities will
%change is limited by the shortcomings of GCMs (General Circulation Models).
%
%This research aims to understand the effects of rainfall-intensity changes
%caused by climate change on soil erosion by water. Firstly, current trends
%of observed rainfall intensity changes are analysed in order to anticipate
%future rainfall intensity changes. Secondly, the rainfall intensity
%information required for soil erosion estimation is determined. Thirdly,
%this research investigates the implications of rainfall intensity changes
%on soil erosion using computerized models. Lastly, this thesis attempts to
%estimate future rates of soil erosion caused by future rainfall intensity
%changes.
%
%Three observed rainfall datasets---Monthly 0.5\textdegree\ grid data, daily
%station data and tipping-bucket rainfall data---were acquired from South
%Downs, UK. One hundred year-long monthly 0.5\textdegree\ grid data were
%analysed to draw outlines of rainfall amount trends in the study area.
%Trends of daily rainfall amount, number of raindays, simple daily intensity
%index, number of raindays with rainfall amount $\geq$10 mm, and number of
%raindays with rainfall amount $\geq$20 mm were also investigated with 9--93
%year-long daily station data. Then, detailed rainfall-intensity patterns in
%the study area were examined using tipping-bucket gauge rainfall data.
%
%Three process-based erosion models, WEPP (Water Erosion Prediction
%Project), EUROSEM (EURopean Soil Erosion Model), and RillGrow were used to
%simulate runoff and soil loss rates with various rainfall intensities.
%Extreme daily rainfall events with highest rainfall intensity and greatest
%rainfall amount were selected from the tipping-bucket rainfall dataset. The
%storm data were aggregated into predefined time steps ranging from one
%minute to 60 minutes. Runoff and soil losses for these temporally varying
%rainfall data were simulated using three erosion models. The results were
%compared to find out the effects of temporal resolution of rainfall data
%used for simulations of runoff and soil loss. This test determines how the
%scale of rainfall data affects results of erosion modelling.
%
%Effects of rainfall-intensity patterns within storms were studied using
%design storms which have increasing, decreasing, peak-in-the-middle and
%constant intensities, respectively. All storms have the same rainfall
%amount. This provides insights into the responses of erosion models to the
%changes of rainfall intensity pattern within a storm. This, in turn, gives
%better understandings of the effects of rainfall intensity changes during a
%storm on soil erosion by water in reality. An additional daily rainfall
%event with ``intra-storm gaps'' was selected. Runoff and soil loss were
%simulated twice using the event data with ``intra-storm gaps'' and without
%the gaps by removing the ``no-rain'' phase. This examines effects of
%``intra-storm gaps'' within single storm event on soil erosion. Limitations
%of current soil erosion models for incorporating future rainfall intensity
%changes were also discussed.
%
%Future soil erosion rates were estimated using WEPP. One hundred year-long
%weather was generated using CLIGEN (Climate Generator designed for WEPP) as
%reference weather data. Future rainfall-intensity changes were anticipated
%by proportionally changing rainfall intensity related parameters in CLIGEN
%input file. No corresponding rainfall amount change is considered. This
%provides limited but vital information on future soil erosion trends which
%are affected by rainfall intensity changes.

%Soil is one of the essentials for life on Earth, together with water, air
%and sunlight. Undoubtedly, soil is an important resource for the survival
%of the human race as most agricultural products are cultivated on soil.
%Sustaining soil conditions is therefore a crucial issue where soil
%nutrients, which control crop yields, fibre and fuel, are scarce.
Existing simulation studies of the effects of future climate change upon erosion
indicate that, under land usages that leave the soil unprotected, even minor
increases in rainfall amounts are likely to result in disproportionately large
increases in erosion, but make the simplifying assumption that distributions of
future rainfall intensities remain unchanged from the present. This research
aims to determine implications of rainfall-intensity changes on soil erosion
using computerised models. Thus, this thesis is a step towards the ultimate goal
of predicting future rates of soil erosion caused by future rainfall intensity
changes.
%
Three soil erosion models, WEPP, EUROSEM, and RillGrow are employed to
investigate impacts of various rainfall intensities on runoff and soil loss
rates. Two extreme daily rainfall events in summer and autumn are subjectively
selected from the tipping-bucket rainfall data, and runoff and soil losses are
simulated using three erosion models. Estimated runoff and soil loss rates with
high resolution rainfall data are greater than those with low temporal
resolution rainfall data. WSIPs (Within-Storm Intensity Patterns) affect soil
erosion amount, although runoff was not much affected. An additional daily
rainfall event with WSGs (Within-Storm Gaps) within a storm is also selected
to highlight effects of intra-storm pause within a storm on soil erosion.
For a given depth of rainfall, events with constant low intensity produced
dramatically less erosion: thus it appears that assuming a constant (or
averaged) intensity throughout a storm does not provide a good representation of
natural rainfall with its continuously varying intensity.
Analyses of outputs from WEPP simulations reveal a problem that WEPP modifies
original rainfall intensity and, thus, simulates erroneous runoff and erosion
rates.
%
Analysis of three observational rainfall datasets (i.e.\ Monthly 0.5\textdegree\
grid data, daily station data and tipping-bucket gauge data) from the South
Downs, UK show increases in frequency of extreme events, and an increasing trend
in daily rainfall intensity for the future. Future soil erosion rates are
estimated using WEPP and CLIGEN (Climate Generator). 30 year-long weather is
generated by CLIGEN. Likely future rainfall frequency and intensity are
anticipated by changing the mean maximum 30 minutes peak intensity. No future
rainfall amount change is assumed. WEPP simulation results suggest that where
mean maximum 30-min peak intensity of the wet months increases soil erosion
increases at a greater rate than runoff.
%
%Several further investigations are suggested on: Measurement of runoff and
%soil loss with discontinuous storms and their implications for soil
%erosion; An investigation on relationships between duration of no-rain
%period and soil erosion; Development of an erosion model that can fulfil
%the requirements suggested in this thesis; An investigation on the
%relationship of intra-storm intensity patterns and data scale (temporal and
%spatial); An investigation on the trend of intra-storm intensity and
%no-rain periods.
\end{abstract}
