%\linenumbers*
\chapter{RESEARCH BACKGROUND AND OBJECTIVES}
\label{sec:RESEARCHOBJECTIVES}

\section{Background and Direction}
\label{sec:ResearchBackgroundandDirection}

\begin{quotation}
\small ``If politics is the art of the possible, research is surely the art of
the soluble. Both are intensely practical-minded affairs.'' (Peter Medawar
(1915-1987) ``The Act of Creation'' in ``New Statesman'', 19 June 1964)
\end{quotation}

Before going into the aim of this research, it seems more appropriate to explain
the scope and direction of this research first. In this way, readers may
understand more clearly the research aim and rationale behind this aim in this
study.

When this research was first begun in 1999, the initial plan was to use the
simulated future rainfall intensity output from RCM as input to one or more
erosion models, in order to improve upon then-current forecasts of future
erosion rates and eventually to develop a method to link RCM results and erosion
models.

However, after a number of pilot simulations with models at an early stage, it
became apparent that RCM rainfall data could not (and still cannot) be used
directly for erosion simulations. In order to use RCM data for the approach
initially proposed, the data should be able to provide rainfall intensity
information on a required temporal resolution (i.e.\ sub-hourly) with an
acceptable level of confidence. We determined that RCM data did not hold
sufficiently detailed (both temporally and spatially) information for it to
be directly used for erosion simulations, and were not
reliable enough for this type of approach yet---and still it appears that
RCM data are not adequate
\citep{nearing2001-229,michael2005-155,o'neal2005-165}.

Thus, another route was taken. As a first approach, in order to obtain
rainfall data usable for erosion modelling in terms of resolution and
representable as future rainfall, observed rainfall data were analysed to
determine trends of rainfall intensity. The idea was that once rainfall
intensity trends had been determined, probable scenarios of future rainfall
intensity changes could then be built by applying the trend onto observed
rainfall intensity.
%More on this approach is discussed in Chapter
%\ref{sec:RainfallCharacteristicsOfTheStudyArea}.

Rather later, problems were identified with selected erosion models, in
that there were aberrant model responses to changes in a certain aspects of
rainfall intensity (See Chapter
\ref{sec:EFFECTSOFCONTINOUSANDDISCONTINUSSTORM}). This implied that there were
deficiencies in the process understanding on which the models are based. Thus,
the original plan had to be put on hold until improved erosion models become
available, and until more reliable future rainfall data (with high temporal
resolution) become available from one source or another.

Accordingly, the focus of the research has evolved into the evaluation of the
response of erosion models to rainfall intensity changes, and implicitly the
process understanding on which these models are based, using arbitrary changes
in rainfall intensity. The idea now is that this will assist in
improving the performance of
erosion models with respect to changes of rainfall intensity by highlighting
where current problem exists. Consequently, greater knowledge here will,
once future changes in rainfall intensity become better known and appropriate
rainfall data become available, improve our ability to estimate future rates of
erosion.
%%
%Favis-Mortlock, D.T., Boardman, J. and MacMillan, V.J.(2001). The limits of
%erosion modeling: why we should proceed with care. In, Harmon, R.S. and Doe
%III, W.W. (eds). Landscape Erosion and Evolution Modeling, Kluwer
%Academic/Plenum Publishing, New York, pp. 477-516.
%%

\section{Objectives and Rationales}
\label{sec:ResearchObjectives}

Thus, the main objective of this research is to investigate:
\begin{itemize}
 \item the implications of change in future rainfall intensity for future soil
erosion by water, by analysing the response of erosion models to arbitrary
rainfall intensity changes, and
 \item implicitly, the process understanding on which the the erosion models are
based.
\end{itemize}
To accomplish these aims, the research was carried out in two parts:
\begin{itemize}
 \item \textit{Rainfall Intensity and Erosion: Model Descriptions and
Responses} and
% \item \textit{ Observed Rainfall Characteristics (and Intensity) of the Study
%Site} and
 \item \textit{Implications for Model-based Studies of Future Climate Change and
Soil Erosion}.
\end{itemize}

In the first part, \textsl{Rainfall Intensity and Erosion: Model Descriptions
and Responses}, three process-based erosion models, WEPP, EUROSEM and RillGrow,
were used to investigate their responses (i.e.\ runoff and soil loss rates) to
various rainfall intensity conditions. The reason for selecting these particular
models is further explained separately in Section \ref{sec:ModelConfiguration}
(See p\pageref{sec:ModelConfiguration}). The process descriptions of these
models were examined in regard to how they represent and make use of rainfall
intensity.

The reason for using multiple models is to minimize the probability of
uncertainty that may increase when relying on a single model
\citep{ipcc2001-881}. The design purposes of erosion models varies from model to
model, and so do their artefacts \citep{favis-mortlock1998-141,jetten1999-521}.
Thus, it is problematic to use only one model, unless there are some
observational data that it can be compared to, as it is not possible to know to
what extent the result from this model is unique to that model
\citep{favis-mortlock1998-141,jetten1999-521}.

For the same reason, it was suggested that, in the study of future climate
change, one should not rely on a single GCM or RCM. This is also the case for
the downscaling \citep{mearns2003,wilby2004}. In such studies, the same
principle---more is better than one---is always in practice \citep{wilby2004}.
This principle may equally be applied to the study of soil erosion modelling.

It would have been possible to use only two erosion models. However, this may
have led to another problematic situation. For example, if two erosion models
show contrasting results, it will be very difficult to decide which one to
accept and which one to accept not---although it may still be debatable
whether the resulting conclusion is applicable to the reality even when both
models agree. A good answer to this dilemma may be found in an old fisherman's
saying: ``Never go to sea with two chronometers; take one or \emph{three}.''
%****** good, but need to be clearer, since this is important
Therefore, \emph{three} erosion models were used in this study.
% in this thesis:\ WEPP,EUROSEM and RillGrow.

Comparing results from three models, instead of one or two models, may increase
our chances to relate modelling results back to the real world.
% provide less uncertainty than comparing results from one or two models.
If all three models show similar responses---even though the chances of
incurring such a result are slim---to rainfall intensity changes, the agreed
results among three models may possibly be related to the real world
\citep{araujo2007-42}.
More importantly, however, when any of the models disagree, further
investigation should follow to look into the model equations, programming
algorithms and codes in order to find out what may have caused such
disagreement. By comparing outputs from these models, we could also identify
their weaknesses and, in turn, improve them for future research.

The primary purpose of these erosion models (i.e. WEPP, EUROSEM and RillGrow) is
of course to simulate the effects of soil erosion by water.
Even though each model has its unique way of representing erosion
processes, the main design purpose is the same; to simulate real-world
erosion processes. However, erosion models are developed for several different
purposes; they may be required to:
\begin{itemize*}
  \item Evaluate the role of the different factors which affect erosion, and
so improve our understanding of erosion processes
  \item Be used as predictive tools for unmonitored (or future) landscapes, in
order to minimise the agricultural and environmental problems caused by soil
erosion.
\end{itemize*}
Because of these purposes, \emph{ideally} all erosion models should be based on
similar understanding of erosion, and simulate erosion similarly. %really???
This general idea was taken into account and it was hypothesised that the
models selected for this research produce similar results when a given
rainfall intensity was applied to them.
Yet, the reality is somewhat different from the ideal, and one may still expect
that the models may produce divergent responses
\citep{favis-mortlock1998-141,jetten1999-521}. However, it is important to
investigate this diversity of model outputs in order to identify and improve
areas where our understanding is limited.
%because of physical difficulties in creating appropriate conditions to study.
%I need to support my argument more here. also, what extent the models are
%different and similar.

\begin{quotation}
\small ``The purpose of models is not to fit the data, but to sharpen the
questions.'' (Samuel Karlin, Eleventh RA Fisher Memorial Lecture, Royal Society,
20 April 1983)
\end{quotation}

The above statement by Karlin highlights one of reasons why many models are used
in numerical and analytical studies. In most cases, a model is based on
knowledge that is limited by what we already know about the process. There can
be a range of different understandings of the same processes---soil erosion
processes in this case. These understandings are expressed as mathematical
equations, and then translated into computing languages, and finally put
together as a model with which our understandings of the processes can be
tested. This provides us (in theory) with a complete control over those input
factors which affect erosion. This is attractive because generally it
is not possible to gain complete control over affecting factors in field
experiments or in laboratory experiments, because modifying only
one factor without altering other factors is not feasible. Using a model, an
individual input parameter may be isolated, adjusted and investigated to find
its effects on the overall erosion process, while keeping all the remaining
parameters constant. Of course, this still does not guarantee a correct
prediction by the model. However, this kind of approach helps to ``sharpen the
questions''. More focused questions from modelling studies may help to fill out,
or rather to pinpoint the gap in our understanding of the processes. There have
been several studies employing this type of approaches which aim to
quantify future erosion
\citep{favis-mortlock1995-365,favis-mortlock1999-329,pruski2002-7,
nearing2005-131}.

One more reason for using a model can be explained by the duration of
experiments. According to \citet{web_[Se-list]}, in the case of erosion-plot
experiments, the outcome may be very difficult to interpret unless the
experiments were conducted for \emph{a long period of time} because of high
random variability (`noise') in the observational data. Therefore, the result
may not be significant unless
the record is of sufficient length, treatments are greatly different, and a
sufficient number of replicates is employed \citep{nearing1999-1829}.
\citet{web_[Se-list]} stated ``\ldots it needs to be understood that these
estimates [which are estimated using short periods of record] can have great
error and that longer term records are needed to refine these estimates.'' He
then suggested to design plots and collect the data ``in such a way as to be
able to use the data in evaluation of models and development of parameters that
can be used to extrapolate results to the much wider climatic record than one
can experience in a few years''.

Hence, using a model can be a good choice of method over observational
experiments in some cases like above---and may well be the only method when
estimating a long-term effect which cannot be observed. For example, in the
study by \citet{favis-mortlock1997-79}, EPIC (Erosion-Productivity Impact
Calculator) was used to reproduce the past erosion processes on a hillslope in
South Downs, UK from 7000 BP to the present day, in order to find out the major
factors influencing past soil erosion. In a case like this, modelling is clearly
the only possible choice. Modelling is also the only choice for the study of
impacts of future climate or land-use changes on soil erosion
\citep{favis-mortlock1995-365,favis-mortlock1999-329,pruski2002-7,
pruski2002-climate,nearing2005-131}.

Therefore, the first part of this study, using WEPP, EUROSEM and RillGrow, aims
to investigate the:
\begin{itemize*}
  \item information regarding rainfall intensity which is needed to simulate
soil erosion,
  \item responses of selected erosion models to various changes in rainfall
intensity,
  \item representation of the effects of rainfall intensity on erosion
within each erosion model,
  \item applicability of these erosion models for future research.
\end{itemize*}

%In the second stage, \textsl{Observed Rainfall Characteristics (and Intensity)
%of the Study Site}, a series of observed high-resolution rainfall data were
%obtained to determine a trend of rainfall intensity changes at the research
%site
%in the South Downs, UK (Figure \ref{fig:DailyRainfallDataSite}). One of the
%reasons for choosing this particular site is because the area has been
%extensively monitored for soil erosion since the late 1970s
%\citep{boardman1995-177,boardman2003-176}, so that data availability of the
%site
%is therefore reasonably great. In addition, there is a well-established
%expertise about the area that can be referenced to the current research
%\citep{boardman1995-177,favis-mortlock1995-365,favis-mortlock1997-79,
%favis-mortlock1998-141,boardman2001-346,boardman2003-176}. The datasets used
%here are temporally and spatially different; monthly 0.5\textdegree\ grid data,
%daily station data and tipping-bucket gauge data.
%
%Moreover, as stated previously, this approach was taken because of the
%important
%issues discovered from RCM data. Probable scenarios of future rainfall
%intensity
%were created based on the \textit{present-day} rainfall intensity trend. These
%present trends were also compared with GCM predicted data. The resulting
%scenarios were later used to make predictions regarding future erosion.

%In short, the second stage aims to
%\begin{inparaenum}[1)]
%  \item provide a solution for the problem that have been identified with RCM
%rainfall data during the initial investigation of this research by analysing
%observed rainfall data in the various scales to determine rainfall intensity
%trends;
%  \item build scenarios of future rainfall intensity based on the
%\textit{present-day} rainfall intensity trend, which may, in turn, be used for
%the final stage of the research.
%\end{inparaenum}
%You imply that you are using present-day climate data to predict future
%erosion. No-one can do that! What you are doing, of course, is to create
%scenarios of future rainfall which are *based on* present-day rainfall (and
%then these are used to make predictions re. future erosion). *I* know that you
%know this, but the text isn't clear. Be as clear as possible: it is easy to
%create confusion in the mind of the reader if you are not completely clear.

In the second part, \textsl{Implications for Model-based Studies of Future
Climate Change and Soil Erosion}, future rainfall intensity scenarios
were used to estimate erosion rates using WEPP, one of the soil
erosion models used in this research. WEPP is used in the second part because it
is a continuous simulation model, which is capable of simulating long-term
erosion taking into account the factors such as the complex overlap of
temporally and spatially diverse distributions of rainfall, erodibility, soil
conditions, plant cover \citep{nearing2006-145}.

The second part of this thesis aims to:
\begin{itemize*}
  \item suggest the best currently-available way of investigating
impacts of rainfall
\textit{intensity} changes on future erosion and
  \item test this method using scenarios of future change in rainfall intensity.
\end{itemize*}

% The aims of all the stages are summarised below:
% \begin{enumerate}
%   \item To investigate the applicability of three models for subsequent
%investigations in terms of their process descriptions on rainfall intensity
%   \item To determine rainfall intensity information requirements for erosion
%simulations
%   \item To determine trends of rainfall intensity at the research site and to
%evaluate the consistency of the trend with GCM predictions \label{aim3}
%   \item To construct future scenarios of rainfall intensity changes based on
%the trend determined from \#\ref{aim3}
%   \item To suggest a method of investigating impacts of rainfall intensity
%changes on future erosion \label{aim4}
%   \item To set out an exemplary case study by using one of the suggested
%method from \#\ref{aim4} with the previously constructed scenarios of rainfall
%intensity change
% \end{enumerate}

\section{Questions To Be Considered}
\label{sec:ResearchQuestions}
This research addresses the following questions:
\begin{enumerate}[{Question} 1.]
  \item What role does rainfall intensity play in the process descriptions which
comprise erosion models?\label{researchquestion1}
  \item Assuming that we use a model to predict erosion rates under a future
climate, and that this future climate has different rainfall intensity from the
present, what information (both regarding climate, and regarding erosion
processes) do we need to make predictions of soil erosion rates under that
future climate? \label{researchquestion2}
%  \item What do we know about future rainfall intensity? Does any trend exist
%in
%the present and past rainfall intensity at the study site? If so, is the trend
%consistent with future rainfall intensity predicted by climate models? If not,
%what must be done to estimate future rainfall intensity?
%\label{researchquestion3}
  \item From the above, are we in a position to predict soil erosion under the
future climate? If not, what more is necessary?\label{researchquestion3}
\end{enumerate}

\section{Outcomes}
\label{sec:ExpectedOutcomes}
Once all the research questions listed above have been answered, the following
outcomes can be attained:
\begin{itemize}
  \item Better understanding of the role of rainfall intensity in soil erosion
model processes,
  \item Information requirements of rainfall intensity for soil erosion
modelling,
  \item Probable estimation of future rainfall intensity, and
  \item Required criteria of rainfall data and erosion models for predicting
future soil erosion.
\end{itemize}

%\nolinenumbers
